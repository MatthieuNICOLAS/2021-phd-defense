\PassOptionsToPackage{unicode}{hyperref}
\PassOptionsToPackage{naturalnames}{hyperref}

\documentclass[10pt]{beamer}

\usetheme[sectionpage=none,block=fill]{metropolis}

\usepackage{acronym} % \ac[p], \acl[ptl], \acs[p], \acf[p]
\usepackage{algorithm, algpseudocode}
\usepackage{appendixnumberbeamer}
\usepackage[backend=biber,defernumbers=true,bibstyle=trad-plain,citestyle=authortitle,sorting=none,maxbibnames=1,maxcitenames=1]{biblatex}
\usepackage{amsmath, amssymb, amsthm}
\usepackage[french]{babel}
\usepackage{bookmark}
\usepackage{booktabs} % \toprule, \midrule, \cmidrule, \bottomrule
\usepackage{cancel} % \cancel
\usepackage{caption}
\usepackage[scale=2]{ccicons}
\usepackage{csquotes} % Dépendance de babel
\usepackage{graphicx}
\hypersetup{hidelinks}
% \usepackage[inline]{enumitem}
% \setlist[enumerate]{label=(\roman*)} %% <- set the base level label separately
\usepackage{marvosym} % \Flatsteel
\usepackage{MnSymbol} % \dashrightarrow
\usepackage{pifont} % \ding
\usepackage{silence} % \WarningFilter
\WarningFilter{biblatex}{Patching footnotes failed}
\usepackage{subcaption} % subfigure
\usepackage{tikz} % \begin{tikzpicture} \end{tikzpicture}
\usetikzlibrary{calc}
\usetikzlibrary{graphs}
\usetikzlibrary{positioning}
\usetikzlibrary{quotes}
\usetikzlibrary{shapes.misc}
\usepackage{wasysym} % \checked
\usepackage{xcolor}
\usepackage{xspace} % \xspace

%-------------------------------------------------------------------
%                           Assets
%-------------------------------------------------------------------
\input{assets/acronyms}
\input{assets/algorithms}
\input{assets/autorefs}
\addbibresource{biblio.bib}
\AtBeginBibliography{\footnotesize}
\setbeamertemplate{bibliography item}[text] % use ref number in bibliography

\input{assets/colors}
\input{assets/figs}
\input{assets/ids}
\input{assets/maths}
\input{assets/misc}
\input{assets/tikz}

\author{
  \textbf{Matthieu Nicolas} (\texttt{matthieu.nicolas@loria.fr})
}
\date{20 décembre 2022}
\title{Ré-identification sans coordination dans les types de données répliquées sans conflits}
% \subtitle{}
\institute{
  \begin{table}
    \resizebox{\columnwidth}{!}{
      \begin{tabular}{lll}
        \emph{Rapporteurs :}  & Hanifa Boucheneb  & Professeure, Polytechnique Montréal\\
                              & Davide Frey       & Chargé de recherche, HdR, Inria Rennes Bretagne-Atlantique \\[0.5em]
        \emph{Examinateurs :} & Hala Skaf-Molli   & Maîtresse de conférences, HdR, Nantes Université, LS2N \\
                              & Stephan Merz      & Directeur de Recherche, Inria Nancy - Grand Est \\[0.5em]
        \emph{Encadrants :}   & Olivier Perrin    & Professeur des Universités, Université de Lorraine, LORIA \\
                              & Gérald Oster      & Maître de conférences, Université de Lorraine, LORIA \\
      \end{tabular}
    }
  \end{table}
  \vspace{1em}
  \resizebox{\columnwidth}{!}{
    \includegraphics[height=1.2em]{img/loria-logo.png}\hspace{1em}
    \includegraphics[height=1.2em]{img/ul-logo.pdf}\hspace{1em}
    \includegraphics[height=1.2em]{img/inria-logo.pdf}\hspace{1em}
    \includegraphics[height=1.2em]{img/cnrs-logo.png}
  }
}

\begin{document}

\begin{frame}[t,plain]
  \maketitle
\end{frame}

\begin{frame}{Applications collaboratives}
    TODO: Voir comment représenter une appli collaboratice à ce stade
    \begin{itemize}
        \item Un \alert{système collaboratif} est un système supportant ses utilisateur-rices dans leurs processus de collaboration pour la réalisation de tâches.
    \end{itemize}
\end{frame}

\begin{frame}{Démocratisation des applications collaboratives}
    TODO: Voir comment illustrer ce point.
    Screenshots et nombre d'utilisateurs en-dessous ?
\end{frame}

\begin{frame}{Avantages d'une architecture basée sur le cloud...}
    TODO: Représenter collaboration via appli basée sur le cloud
    \begin{itemize}
        \item \alert{Disponibilité} : Répond aux utilisateur-rices
        \item \alert{Tolérance aux pannes} : Fonctionne malgré pannes
        \item \alert{Capacité de passage à l'échelle} : Supporte activité massive
    \end{itemize}
\end{frame}

\begin{frame}{... et ses limites}
    TODO: Illustrer chacune des propriétés
    \begin{columns}
        \begin{column}{0.5\textwidth}
            \begin{itemize}
                \item Confidentialité
                \item Souveraineté
            \end{itemize}
        \end{column}
        \begin{column}{0.5\textwidth}
            \begin{itemize}
                \item Pérennité
                \item Résistance à la censure
            \end{itemize}
        \end{column}
    \end{columns}
\end{frame}

\begin{frame}[standout]
    Pouvons-nous concevoir des applications collaboratives satisfaisant l'ensemble de ces propriétés ?
\end{frame}

\begin{frame}{Applications collaboratives pair-à-pair \footcite{localfirstsoftware2019}}
    TODO: Illustrer une appli P2P
    \begin{block}{Problématiques}
        En l'absence d'autorités centrales, comment
        \begin{itemize}
            \item résoudre les conflits de modifications ?
            \item authentifier les utilisateur-rices ?
            \item sécuriser les communications ?
        \end{itemize}
    \end{block}
\end{frame}


\begin{frame}{MUTE \singlefootnote{Disponible à : \url{https://mutehost.loria.fr}}}
    \begin{figure}
        \resizebox{0.8 \textwidth}{!}{
            \includegraphics{img/screenshot-mute-editor.png}
        }
    \end{figure}
    \begin{itemize}
        \item Éditeur de texte collaboratif P2P temps réel chiffré de bout en bout
        \item Permet à l'équipe d'étudier et contribuer sur les problématiques des applications \ac{LFS}
    \end{itemize}
\end{frame}

\begin{frame}{Mes contributions}
    TODO: À retravailler pour faire apparaître les problématiques ?
    TODO: Ajouter comparaison des modèles de sync et approches pour CRDTs pour le type Séquence ?
    \begin{itemize}
        \item Conception d'un nouveau \ac{CRDT} pour le type Séquence
        \item Implémentation et intégration de \acp{CRDT} dans MUTE
    \end{itemize}
\end{frame}

\begin{frame}{LogootSplit \footfullcite{2013-logootsplit}}
  \begin{itemize}
    \item State of the art of \emph{Sequence \acp{CRDT}}
    \item Elements are ordered by their identifier, noted here with the following formalism: $\id{position}{node\_id~node\_seq}{offset}$
  \end{itemize}

  \pause

  \begin{columns}
    \begin{column}{0.33\textwidth}
      \begin{figure}
        \centering
        \begin{tikzpicture}
            \path
                node[letter, label=below:{$\id{i}{B0}{0}$}] {H}
                to ++(0:\widthletter) node[letter, label=below:{$\id{i}{B0}{1}$}] {L}
                to ++(0:\widthletter) node[letter, label=below:{$\id{i}{B0}{2}$}] {O};
        \end{tikzpicture}
        \caption{State of a sequence which contains the elements "HLO" and their corresponding identifiers}
      \end{figure}
    \end{column}
    \pause
    \begin{column}{0.33\textwidth}
      \vspace{-9mm}
      \begin{figure}
        \centering
        \begin{tikzpicture}
            \path
                node[block, label=below:{$\id{i}{B0}{0..2}$}] {HLO};
        \end{tikzpicture}
        \caption{State of a sequence which contains the block "HLO"}
      \end{figure}
    \end{column}
    \pause
    \begin{column}{0.33\textwidth}
      \begin{figure}
        \vspace{-7mm}
        \centering
        \begin{tikzpicture}
            \path
              node[letter, label=below:{$\id{i}{B0}{0}$}] {H}
              to ++(0:\widthletter) node[letter, fill=\colorblockone, label=above:{\coloridone$\id{i}{B0}{0}\id{f}{A0}{0}$}] {E}
            to ++(0:\widthletter) node[block, label={below:$\id{i}{B0}{1..2}$}] {LO};
        \end{tikzpicture}
        \caption{State of a sequence which contains the elements "HELO" and their corresponding identifiers}
      \end{figure}
    \end{column}
  \end{columns}
\end{frame}


\begin{frame}{Identifier constraints}
  \begin{itemize}
    \item To fulfill their role, identifiers have to comply to several constraints:
  \end{itemize}

  \begin{block}{Globally unique}
    \begin{itemize}
      \item Identifiers should never be generated twice, neither by different users nor by the same one at different times
    \end{itemize}
  \end{block}
  \begin{block}{Totally ordered}
    \begin{itemize}
      \item We should always be able to compare and order two elements using their identifiers
    \end{itemize}
  \end{block}
  \begin{block}{Dense set}
    \begin{itemize}
      \item We should always be able to add a new element, and thus a new identifier, between two others
    \end{itemize}
  \end{block}
\end{frame}

\begin{frame}{LogootSplit identifiers}
  \begin{itemize}
    \item To comply with these constraints, LogootSplit proposes identifiers composed of quadruplets of integers of the following form:
  \end{itemize}
  \begin{center}
    $\id{position}{node\_id~node\_seq}{offset}$
  \end{center}
  \begin{itemize}
    \item \emph{position} allows to determine the position of this identifier compared to others
    \item \emph{node\_id} refers to the node's identifier, assumed to be unique
    \item \emph{node\_seq} refers to the node's logical clock, which increases monotonically with local operations
    \item \emph{offset} refers to the element position in its original block
  \end{itemize}
\end{frame}

\begin{frame}{Research issue}

  \begin{itemize}
    \item \textbf{Evergrowing overhead:} impacts memory, bandwidth and CPU
  \end{itemize}

  \begin{columns}
    \begin{column}{0.45\textwidth}
      \begin{figure}
        \centering
        \includegraphics[width=\textwidth]{img/overhead-size.pdf}
        \caption{Memory footprint of the data structure}
      \end{figure}
    \end{column}
    \begin{column}{0.55\textwidth}
      \begin{itemize}
        \item \textbf{Operation count:} 100k
        \item \textbf{Size of content:} 100KB
        \item \textbf{Size of data structure:} 20MB
      \end{itemize}
    \end{column}
  \end{columns}

  \centering
  \alert{How to reduce the overhead introduced by the data structure?}
\end{frame}

\begin{frame}{Related work}
  \begin{itemize}
    \item Core-nebula approach \footfullcite{zawirski:hal-01248197}
    \begin{itemize}
      \item Reassigns shorter identifiers to elements\dots
      \item \dots but requires consensus
    \end{itemize}
    \item LSEQ \footfullcite{lseq2017}
    \begin{itemize}
      \item Set of strategies to reduce the growth of identifiers \dots
      \item \dots but overhead still proportional to number of elements
    \end{itemize}
  \end{itemize}
\end{frame}


\section{RenamableLogootSplit}

\input{content/rls-rename}

\subsection{Gestion des opérations \ins et \rmv concurrentes}

% \begin{frame}{Handling concurrent operations}
  \begin{figure}
    \centering
    \includegraphics<1>[width=\columnwidth]{../2021-phd-day-figures/inconsistency-concurrent-op-to-rename/1/figure.pdf}
    \includegraphics<2>[width=\columnwidth]{../2021-phd-day-figures/inconsistency-concurrent-op-to-rename/2/figure.pdf}
    \includegraphics<3>[width=\columnwidth]{../2021-phd-day-figures/inconsistency-concurrent-op-to-rename/3/figure.pdf}
    \includegraphics<4>[width=\columnwidth]{../2021-phd-day-figures/inconsistency-concurrent-op-to-rename/4/figure.pdf}
    \caption{Applying naively concurrent update}
  \end{figure}
  \vspace{-3mm}
  \begin{itemize}
    \item <2->Can issue operations concurrently to \emph{rename}
    \item <4>Produce inconsistencies if applied naively
  \end{itemize}
\end{frame}

\begin{frame}{Fixing handling concurrent operations}
  \begin{figure}
    \centering
    \includegraphics<1>[width=\columnwidth]{../2021-phd-day-figures/handling-concurrent-op-to-rename/1/figure.pdf}
    \includegraphics<2>[width=\columnwidth]{../2021-phd-day-figures/handling-concurrent-op-to-rename/2/figure.pdf}
    \includegraphics<3-6>[width=\columnwidth]{../2021-phd-day-figures/handling-concurrent-op-to-rename/3/figure.pdf}
    \includegraphics<7>[width=\columnwidth]{../2021-phd-day-figures/handling-concurrent-op-to-rename/4/figure.pdf}
    \includegraphics<8>[width=\columnwidth]{../2021-phd-day-figures/handling-concurrent-op-to-rename/5/figure.pdf}
    \caption{Handling properly concurrent update}
  \end{figure}
  \vspace{-3mm}
  \begin{itemize}
    \item<1-> Use \emph{epoch-based} system to track concurrent operations
    \item<4-> Transform operations against \emph{rename} ones (\emph{OT})
    \begin{enumerate}
      \item<5-> Find predecessor in former state ($\id{i}{B0}{0}\id{f}{A0}{0}$)
      \item<6-> Find its counterpart in new state ($\id{i}{A1}{1}$)
      \item<7-> Prepend it to given id to form new id ($\id{i}{A1}{1}\id{i}{B0}{0}\id{m}{B1}{0}$)
    \end{enumerate}
  \end{itemize}
\end{frame}


\subsection{Gestion des opérations \ren concurrentes}

% \begin{frame}[standout]
  What about concurrent \emph{rename} operations ?
\end{frame}

\begin{frame}{What about concurrent \emph{rename} operations ?}
  \begin{figure}
    \centering
    \includegraphics<1>[width=\columnwidth]{../2021-phd-day-figures/divergent-concurrent-rename/1/figure.pdf}
    \includegraphics<2->[width=\columnwidth]{../2021-phd-day-figures/divergent-concurrent-rename/2/figure.pdf}
    \caption{Concurrent \emph{rename} operations leading to divergent states}
  \end{figure}
  \vspace{-3mm}
  \begin{itemize}
    \item<3-> \emph{Rename} operations are system operations
    \item<4> Can resolve conflict by only applying one of them
  \end{itemize}
\end{frame}

\begin{frame}{How to do so ?}
  \begin{figure}
    \centering
    \includegraphics<1>[width=0.3\columnwidth]{../2021-phd-day-figures/epoch-tree/1/figure.pdf}
    \includegraphics<2->[width=0.3\columnwidth]{../2021-phd-day-figures/epoch-tree/2/figure.pdf}
    \caption{\emph{Epoch tree} corresponding to previous scenario}
  \end{figure}
  \vspace{-3mm}
  \begin{itemize}
    \item<2-> Have to pick an epoch as the {\color{red} target one}
    \begin{itemize}
      \item<2-> Define total order on epochs
    \end{itemize}
    \item<3> Have to move through the tree
    \begin{itemize}
      \item<3> Design transformation function to revert \emph{rename} operation
    \end{itemize}
  \end{itemize}
\end{frame}

\begin{frame}{Applying concurrent \emph{rename} operations}
  \begin{figure}
    \centering
    \includegraphics<1>[width=\columnwidth]{../2021-phd-day-figures/resolving-concurrent-rename/1/figure.pdf}
    \includegraphics<2>[width=\columnwidth]{../2021-phd-day-figures/resolving-concurrent-rename/2/figure.pdf}
    \includegraphics<3>[width=\columnwidth]{../2021-phd-day-figures/resolving-concurrent-rename/3/figure.pdf}
    \caption{Applying a concurrent \emph{rename} operation}
  \end{figure}
  \vspace{-3mm}
  \begin{itemize}
    \item<2-> Revert state to equivalent one at LCA epoch
    \item<3> Apply then \emph{rename} operations leading to target epoch
  \end{itemize}
\end{frame}

\begin{frame}{Downsides}
  \begin{block}{Need to store former state until no more concurrent operations}
    \begin{itemize}
      \item Can garbage collect it once the \emph{rename} operation is causally stable \footfullcite{10.1007/978-3-662-43352-2_11}
      \item Can offload it to the disk meanwhile
    \end{itemize}
  \end{block}

  \begin{block}{Need to propagate former state to other nodes}
    \begin{itemize}
      \item Can compress the operation to minimise bandwidth consumption
      \item Can trigger \emph{rename} operations at a given number of blocks
    \end{itemize}
  \end{block}
\end{frame}


\subsection{Mécanisme de GC des époques obsolètes}

% \subsection{Mécanisme de GC des époques obsolètes}


\subsection{Évaluation}

% \begin{frame}[standout]
  \alert{Evaluation}

  \bigskip
  Ran simulations to compare performance of RenamableLogootSplit to LogootSplit one
\end{frame}

\begin{frame}{Scenario}
  \begin{itemize}
    \item Simulate collaborative editing sessions using either LogootSplit or RenamableLogootSplit
    \item \textbf{Phase 1 (content generation):} 80/20\% of \emph{insert}/\emph{remove}
    \item \textbf{Phase 2 (editing):} 50/50\% of \emph{insert}/\emph{remove}
    \item Nodes switch to phase 2 when document reaches critical size (15 pages - 60k elements)
  \end{itemize}
\end{frame}

\begin{frame}{Experimental settings}
  \begin{itemize}
    \item Use Node.js version 13.1.0
    \item Obtained documents sizes using our fork of \emph{object-sizeof} \footnote{\url{https://www.npmjs.com/package/object-sizeof}}
    \item Ran benchmarks on a workstation equipped of a Intel Xeon CPU E5-1620 (10MB Cache, 3.50 GHz) with 16GB of RAM running Fedora 31
    \item Measured times using \texttt{process.hrtime.bigint()}
  \end{itemize}
\end{frame}

\begin{frame}{Results - Convergence}
  \begin{itemize}
    \item Compared final content of nodes per sessions
    \item Did not observe any divergence
    \bigskip
    \item Empirical result, not a proof...
    \item ... but represents first step towards the validation
  \end{itemize}
\end{frame}

\begin{frame}{Results - Memory footprint}
  \begin{figure}
    \centering
    \only<1>{\hspace{-10mm}\includegraphics[width=0.61\textwidth]{img/snapshots-sizes-1.pdf}}
    \includegraphics<2>[width=0.7\textwidth]{img/snapshots-sizes-2.pdf}
    \includegraphics<3>[width=0.7\textwidth]{img/snapshots-sizes.pdf}
    \caption{Evolution of the size of the document}
    \label{fig:evolution-document-size}
  \end{figure}

  \vspace{-1\baselineskip}
  \begin{itemize}
    \item<2-> \emph{Rename} resets the overhead of the CRDT, if can garbage collect
    \item<3> \emph{Rename} still reduces by $66\%$ the size otherwise
  \end{itemize}
\end{frame}

\begin{frame}{Results - Integration time of \emph{insert} operations}
  \begin{figure}
    \captionsetup[subfigure]{aboveskip=-1pt,belowskip=-1pt}
    \centering
    \begin{subfigure}{0.47\columnwidth}
        \includegraphics[width=1\textwidth]{img/integration-time-boxplot-local-operations-without-outliers.pdf}
        \caption{Local operations}
        \label{fig:evolution-integration-time-local-insert-remove}
    \end{subfigure}
    \begin{subfigure}{0.47\columnwidth}
        \includegraphics[width=1\textwidth]{img/integration-time-boxplot-remote-operations-without-outliers.pdf}
        \caption{Remote operations}
        \label{fig:evolution-integration-time-remote-insert-remove}
    \end{subfigure}
    \caption{Evolution of the integration time of \emph{insert} operations}
    \label{fig:evolution-integration-time-insert-remove}
  \end{figure}

  \vspace{-1\baselineskip}
  \begin{itemize}
    \item<2-> \emph{Rename} reduces integration times of future operations
    \item<3> Transforming concurrent operations is actually faster than applying them on former state
  \end{itemize}

\end{frame}

\begin{frame}{Results - Integration time of \emph{rename} operations}
  \begin{table}[!ht]
      \centering
      \resizebox{\columnwidth}{!}{
          \begin{tabular}{lrrrrr}
              \toprule
              \multicolumn{2}{c}{Parameters} & \multicolumn{4}{c}{Integration Time (ms)} \\
              \cmidrule(lr){1-2} \cmidrule(lr){3-6}
              Type & Nb Ops (k) &   Mean &   Median & 99\textsuperscript{th} Quant. &    Std \\
              \midrule
              Local & 30  &    41.75 &    38.74 &      71.68 &   6.84 \\
                                      & 90  &   119.19 &   118.87 &     124.22 &   2.49 \\
                                      & 150 &   158.04 &   157.95 &     164.38 &   2.49 \\
              \cmidrule(lr){1-6}
              Remote & 30  &   481.32 &   477.13 &     537.30 &  17.11 \\
                                      & 90  &  1491.28 &  1481.83 &    1657.58 &  51.10 \\
                                      & 150 &  1694.17 &  1675.95 &    1852.55 &  59.94 \\
              \bottomrule
          \end{tabular}
      }
      \caption{Integration time of rename operations}
  \end{table}

  \begin{itemize}
    \item<2-> Noticeable by users
    \item<3> Need to improve remote integration time
  \end{itemize}
\end{frame}

\begin{frame}{Results - Integration time of \emph{rename} operations (complete)}
  \begin{table}[!ht]
      \centering
      \resizebox{\columnwidth}{!}{
          \begin{tabular}{lrrrrr}
              \toprule
              \multicolumn{2}{c}{Parameters} & \multicolumn{4}{c}{Integration Time (ms)} \\
              \cmidrule(lr){1-2} \cmidrule(lr){3-6}
              Type & Nb Ops (k) &   Mean &   Median & 99\textsuperscript{th} Quant. &    Std \\
              \midrule
              Local & 30  &    41.75 &    38.74 &      71.68 &   6.84 \\
                                      & 90  &   119.19 &   118.87 &     124.22 &   2.49 \\
                                      & 150 &   158.04 &   157.95 &     164.38 &   2.49 \\
              \cmidrule(lr){1-6}
              Direct remote & 30  &   481.32 &   477.13 &     537.30 &  17.11 \\
                                      & 90  &  1491.28 &  1481.83 &    1657.58 &  51.10 \\
                                      & 150 &  1694.17 &  1675.95 &    1852.55 &  59.94 \\
              \cmidrule(lr){1-6}
              Cc. int. greater epoch & 30  &   643.53 &   643.57 &     682.80 &  13.42 \\
                                      & 90  &  1998.23 &  1994.08 &    2111.98 &  45.37 \\
                                      & 150 &  2241.92 &  2233.61 &    2351.02 &  52.20 \\
              \cmidrule(lr){1-6}
              Cc. int. lesser epoch & 30  &     1.36 &     1.30 &       3.53 &   0.37 \\
                                      & 90  &     4.45 &     4.23 &       5.81 &   0.71 \\
                                      & 150 &     5.53 &     5.26 &       8.70 &   0.79 \\
              \bottomrule
          \end{tabular}
      }
      \caption{Integration time of rename operations}
  \end{table}
\end{frame}


\subsection{Conclusion}

% \begin{frame}{Conclusion}
  \begin{block}{Done}
    \vspace{-1mm}
    \begin{itemize}
      \item Designed a new Sequence \ac{CRDT} : RenamableLogootSplit
      \item Validated it through experimental evaluation
    \end{itemize}
  \end{block}

  \pause

  \begin{block}{Work in progress}
    \vspace{-1mm}
    \begin{itemize}
      \item Publishing it
      \item Writing the manuscript
    \end{itemize}
  \end{block}

  \pause

  \begin{block}{To do}
    \vspace{-1mm}
    \begin{itemize}
      \item Prove formally the correctness of the mechanism
      \item Design better strategies to select the target epoch
      \item Improve performance of \emph{rename} operations
    \end{itemize}
  \end{block}
\end{frame}


\section{Conclusion et perspectives}

% \metroset{sectionpage=progressbar}
\section{Conclusion générale \& Perspectives}
\metroset{sectionpage=none}

\begin{frame}{Conclusion}
  \metroset{block=transparent}
  \begin{block}{Contributions}
    \begin{itemize}
      \item Mécanisme de renommage pour CRDTs pour le type Séquence à identifiants densément ordonnés
      \item Implémentation de RenamableLogootSplit et de ses dépendances (protocole d'appartenance au réseau, couche de livraison) dans MUTE
      \item Comparaison des différents modèles de synchronisation pour CRDTs, ainsi que
      \item Comparaison des différentes approches pour CRDTs pour le type Séquence
    \end{itemize}
  \end{block}
\end{frame}

\begin{frame}{Perspectives}
  \metroset{block=transparent}
  \begin{block}{RenamableLogootSplit}
    \begin{itemize}
      \item Peut-on définir une relation \emph{priority} $\lepoch$ réduisant calculs à échelle du système ?
    \end{itemize}
  \end{block}
  \pause
  \begin{block}{CRDTs}
    \begin{itemize}
      \item Doit-on encore concevoir CRDTs synchronisés par états ou opérations ?
      \item Peut-on proposer un framework pour conception de CRDTs synchronisés par opérations ?
    \end{itemize}
  \end{block}
\end{frame}

\begin{frame}{Ouverture : Doit-on encore concevoir CRDTs synchronisés par états ou opérations ?}
  \begin{table}[!ht]
    \resizebox{\columnwidth}{!}{
      \begin{tabular}{lccc}
        \toprule
                                                  & Sync. par états & Sync. par opérations    & Sync. par diff. d'états \\
        \midrule
        Forme un sup-demi-treillis                  & \checkmark  & \checkmark  & \checkmark  \\
        Intègre modifications par fusion d'états  & \checkmark  & \ballotx    & \checkmark  \\
        Intègre modifications par élts irréductibles & \ballotx    & \checkmark  & \checkmark  \\
        Résiste nativ. aux défaillances réseau           & \checkmark  & \ballotx    & \checkmark  \\
        Adapté pour systèmes temps réel           & \ballotx    & \checkmark  & \checkmark  \\
        Offre nativ. modèle de cohérence causale             & \checkmark  & \ballotx    & \ballotx  \\
        \bottomrule
      \end{tabular}
    }
  \end{table}
  \begin{itemize}
    \item Synchronisation par différences offre meilleur des mondes\dots
    \item \dots y a-t-il encore un intérêt aux autres modèles, \eg pour composition ou sécurité ?
  \end{itemize}
\end{frame}

\begin{frame}[standout]
  Merci de votre attention, avez-vous des questions ?
  \vspace{3em}
  \begin{center}
    \ccby
  \end{center}
\end{frame}


\begin{frame}[standout]
  Thanks for your attention, any questions?
  \vspace{3em}
  \begin{center}
    \ccby
  \end{center}
\end{frame}

\appendix

\end{document}
