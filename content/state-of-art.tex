\section{État de l'art des CRDTs pour le type Séquence}

\begin{frame}{Conflict-free Replicated Data Types (CRDTs) \footcite{shapiro_2011_crdt}}
    \begin{itemize}
        \item Nouvelles spécifications des types de données, \eg \emph{Ensemble} ou \emph{Séquence}
        \item Incorpore nativement mécanisme de résolution de conflits
    \end{itemize}
    \pause
    \begin{block}{Propriétés des CRDTs}
        \begin{itemize}
            \item Permettent modifications \alert{sans coordination}
            \item Garantissent la \alert{convergence forte}
        \end{itemize}
    \end{block}
    \pause
    \begin{block}{Convergence forte}
        Ensemble des noeuds ayant intégrés le même ensemble de modifications obtient des états équivalents, sans nécessiter d'actions ou messages supplémentaires
    \end{block}
\end{frame}

\begin{frame}[fragile]{CRDTs pour le type Séquence}
    \metroset{block=transparent}
    \begin{columns}
        \begin{column}{0.5\textwidth}
            \begin{block}{Type Séquence séquentiel}
                \begin{figure}[!ht]
                    \centering
                    \resizebox{\columnwidth}{!}{
                        \begin{tikzpicture}
                            \newcommand\initialstate[2]{
                                \path
                                    #1
                                    ++#2
                                    node[letter, label=below:{$0$}] {B}
                                    ++(0:\widthletter) node[letter, label=below:{$1$}] {N}
                                    ++(0:\widthletter) node[letter, label=below:{$2$}] {J}
                                    ++(0:\widthletter) node[letter, label=below:{$3$}] {O};
                            }
                            \newcommand\totoa[2]{
                                \path
                                    #1
                                    ++#2
                                    node[letter, label=below:{$0$}] {}
                                    ++(0:\widthletter) node[letter, label=below:{$1$}] {}
                                    ++(0:\widthletter) node[letter, label=below:{$2$}] {}
                                    ++(0:\widthletter) node[letter, label=below:{$3$}] {}
                                    ++(0:\widthletter) node[letter, label=below:{$4$}] {};
                            }
                            \newcommand\totob[2]{
                                \path
                                    #1
                                    ++#2
                                    node[letter, label=below:{$0$}] {B}
                                    ++(0:\widthletter) node[letter, label=below:{$1$}] {}
                                    ++(0:\widthletter) node[letter, label=below:{$2$}] {N}
                                    ++(0:\widthletter) node[letter, label=below:{$3$}] {J}
                                    ++(0:\widthletter) node[letter, label=below:{$4$}] {O};
                            }
                            \newcommand\totoc[2]{
                                \path
                                    #1
                                    ++#2
                                    node[letter, label=below:{$0$}] {B}
                                    ++(0:\widthletter) node[letter, label=below:{$1$}] {A}
                                    ++(0:\widthletter) node[letter, label=below:{$2$}] {N}
                                    ++(0:\widthletter) node[letter, label=below:{$3$}] {J}
                                    ++(0:\widthletter) node[letter, label=below:{$4$}] {O};
                            }

                            \newcommand\offseta{ (90:1) }

                            \path
                                node {\textbf{A}}
                                ++(0:0.5) node (a) {}
                                +(0:10) node (a-end) {}
                                +(0:1) node[point] (a-initial) {}
                                +(0:6) node (a-ins-a) {}
                                +(0:9) node (a-final) {};

                            \initialstate{(a-initial)}{\offseta};
                            \draw[dotted] (a) -- (a-initial) (a-final) -- (a-end);
                            \only<1>{
                                \draw[->, thick] (a-initial) -- (a-final);
                            }
                            \onslide<2->{
                                \path (a-ins-a) node[point, label=-170:{$\trm{ins}(1, A)$}] {};
                                \draw[->, thick] (a-initial) -- (a-ins-a) -- (a-final);
                            }
                            \only<3>{\totoa{(a-ins-a)}{\offseta}}
                            \only<4>{\totob{(a-ins-a)}{\offseta}}
                            \onslide<5->{\totoc{(a-ins-a)}{\offseta}}
                        \end{tikzpicture}
                    }
                \end{figure}
            \end{block}
        \end{column}
        \begin{column}{0.5\textwidth}
            \onslide<7->{
                \begin{block}{CRDTs pour Séquence}
                    \begin{figure}[!ht]
                        \centering
                        \resizebox{\columnwidth}{!}{
                            \begin{tikzpicture}
                                \newcommand\initialstate[2]{
                                    \path
                                        #1
                                        ++#2
                                        node[letter, label=below:{$\trm{id}_0$}] {B}
                                        ++(0:\widthletter) node[letter, label=below:{$\trm{id}_1$}] {N}
                                        ++(0:\widthletter) node[letter, label=below:{$\trm{id}_2$}] {J}
                                        ++(0:\widthletter) node[letter, label=below:{$\trm{id}_3$}] {O};
                                }
                                \newcommand\totoa[2]{
                                    \path
                                        #1
                                        ++#2
                                        node[letter, label=below:{}] {}
                                        ++(0:\widthletter) node[letter, label=below:{}] {}
                                        ++(0:\widthletter) node[letter, label=below:{}] {}
                                        ++(0:\widthletter) node[letter, label=below:{}] {}
                                        ++(0:\widthletter) node[letter, label=below:{}] {};
                                }
                                \newcommand\totob[2]{
                                    \path
                                        #1
                                        ++#2
                                        node[letter, label=below:{$\trm{id}_0$}] {B}
                                        ++(0:\widthletter) node[letter, label=below:{$?$}] {}
                                        ++(0:\widthletter) node[letter, label=below:{$\trm{id}_1$}] {N}
                                        ++(0:\widthletter) node[letter, label=below:{$\trm{id}_2$}] {J}
                                        ++(0:\widthletter) node[letter, label=below:{$\trm{id}_3$}] {O};
                                }
                                \newcommand\totoc[2]{
                                    \path
                                        #1
                                        ++#2
                                        node[letter, label=below:{$\trm{id}_0$}] {B}
                                        ++(0:\widthletter) node[letter, label=below:{$\trm{id}_{0.5}$}] {A}
                                        ++(0:\widthletter) node[letter, label=below:{$\trm{id}_1$}] {N}
                                        ++(0:\widthletter) node[letter, label=below:{$\trm{id}_2$}] {J}
                                        ++(0:\widthletter) node[letter, label=below:{$\trm{id}_3$}] {O};
                                }

                                \newcommand\offseta{ (90:1) }

                                \path
                                    node {\textbf{A}}
                                    ++(0:0.5) node (a) {}
                                    +(0:10) node (a-end) {}
                                    +(0:1) node[point] (a-initial) {}
                                    +(0:6) node (a-ins-a) {}
                                    +(0:9) node (a-final) {};

                                \initialstate{(a-initial)}{\offseta};
                                \draw[dotted] (a) -- (a-initial) (a-final) -- (a-end);
                                \only<7,8>{
                                    \draw[->, thick] (a-initial) -- (a-final);
                                }
                                \onslide<9->{
                                    \path (a-ins-a) node[point, label=-170:{$\trm{ins}(B \prec A \prec N)$}] {};
                                    \draw[->, thick] (a-initial) -- (a-ins-a) -- (a-final);
                                }
                                \only<9>{\totoa{(a-ins-a)}{\offseta}}
                                \only<10>{\totob{(a-ins-a)}{\offseta}}
                                \onslide<11->{\totoc{(a-ins-a)}{\offseta}}
                            \end{tikzpicture}
                        }
                    \end{figure}
                \end{block}
            }
        \end{column}
    \end{columns}
    \begin{itemize}
        \item<6-> Changements des indices est \alert{source de conflits}
        \item<7-> Assignent des \alert{identifiants de position}\footcite{2009-treedoc-preguica} à chaque élément
        \item<7->
            Permettent d'\alert{ordonner les élements}
            \onslide<8->{
                \begin{equation*}
                    \trm{id}_0 \lid \trm{id}_1 \lid \trm{id}_2 \lid \trm{id}_3
                \end{equation*}
            }
        \vspace{-1.5em}
        \item<10->
            Nécessaire que les identifiants appartiennent à un \alert{espace dense}
            \onslide<11->{
                \begin{equation*}
                    \trm{id}_0 \lid \trm{id}_{0.5} \lid \trm{id}_1
                \end{equation*}
            }
    \end{itemize}
    \vspace{-2em}
    \onslide<12->{
        \begin{center}
            \alert{Utilise LogootSplit\footcite{2013-logootsplit} comme base}
        \end{center}
    }
\end{frame}

\begin{frame}{Identifiant LogootSplit}
    \metroset{block=transparent}
    \begin{block}{Identifiant}
        \begin{itemize}
            \item Composé d'un ou plusieurs tuples de la forme
        \end{itemize}
        \vspace{3em}
        \begin{equation*}
            \tikzmarknode{pos}{pos}^{
                \tikzmarknode{nodeId}{nodeId}~\tikzmarknode{nodeSeq}{nodeSeq}
            }_{\tikzmarknode{offset}{offset}}
        \end{equation*}
        \begin{tikzpicture}[overlay,remember picture,>=stealth,nodes={align=left,inner ysep=1pt},<-]
            % For "pos"
            \path<2-> (pos.north) ++ (0,2em) node[anchor=south east,color=ucl1mdred] (legend-pos){\textbf{position (a-z)}};
            \draw<2-> [color=ucl1mdred](pos.north) |- ([xshift=-0.3ex,color=ucl1mdred] legend-pos.south west);
            % For "nodeId"
            \path<3-> (nodeId.north) ++ (0,2em) node[anchor=south west,color=ucl1mdblue] (legend-nodeid){\textbf{identifiant (A-Z) de l'auteur}};
            \draw<3-> [color=ucl1mdblue](nodeId.north) |- ([xshift=0.3ex,color=ucl1mdblue] legend-nodeid.south east);
            % For "nodeSeq"
            \path<4-> (nodeSeq.south) ++ (0,-2em) node[anchor=north west,color=ucl2dkpurple] (legend-nodeseq){\textbf{numéro de séquence (1-9)}};
            \draw<4-> [color=ucl2dkpurple](nodeSeq.south) |- ([xshift=0.3ex,color=ucl2dkpurple] legend-nodeseq.south east);
        \end{tikzpicture}
    \end{block}
    \onslide<5->{
    \begin{block}{Exemples}
            \begin{equation*}
                \id{d}{F5}{0} \onslide<6->{\lid \id{m}{C1}{0}} \onslide<7->{\lid \id{m}{C1}{0}\id{f}{E1}{0}}
            \end{equation*}
        \onslide<8->{
            \begin{equation*}
                \id{i}{B1}{0} \lid \only<8>{\quad?\quad} \only<9>{\id{i}{B1}{0}\id{f}{A1}{0}} \lid \id{i}{B1}{1}
            \end{equation*}
        }
    }
    \end{block}
\end{frame}

\begin{frame}[fragile]{Bloc LogootSplit}
    \begin{itemize}
        \item Coûteux de stocker les identifiants de chaque élément
    \end{itemize}
    \begin{figure}[!ht]
        \begin{tikzpicture}
            \path
                node[letter, label=below:{$\id{m}{C1}{0}$}] {B}
                ++(0:\widthletter) node[letter, label=below:{$\id{m}{C1}{1}$}] {A}
                ++(0:\widthletter) node[letter, label=below:{$\id{m}{C1}{2}$}] {N}
                ++(0:\widthletter) node[letter, label=below:{$\id{m}{C1}{3}$}] {J}
                ++(0:\widthletter) node[letter, label=below:{$\id{m}{C1}{4}$}] {O};
        \end{tikzpicture}
    \end{figure}
    \pause
    \begin{itemize}
        \item Aggrège en un \alert{bloc} éléments ayant \alert{identifiants contigus}
    \end{itemize}
    \begin{block}{Identifiants contigus}
        Deux identifiants sont contigus si et seulement si les deux identifiants sont identiques à l'exception de leur dernier offset et que leur derniers offsets sont consécutifs.
    \end{block}
    \pause
    \begin{itemize}
        \item Note l'intervalle d'identifiants d'un bloc : $\id{pos}{nodeId~nodeSeq}{begin..end}$
    \end{itemize}
    \begin{figure}[!ht]
        \begin{tikzpicture}
            \path
                node[block, label=below:{$\id{m}{C1}{0..4}$}] {BANJO};
        \end{tikzpicture}
    \end{figure}
\end{frame}

\begin{frame}[fragile]{Exemple insertions concurrentes}
    \begin{figure}[!ht]
        \centering
        \resizebox{\columnwidth}{!}{
          \begin{tikzpicture}
            \newcommand\nodehl[1]{
                node[block, label=#1:{$\id{i}{B1}{0..1}$}] {HL}
            }
            \newcommand\nodehlo[1]{
                node[block, label=#1:{$\id{i}{B1}{0..2}$}] {HLO}
            }
            \newcommand\nodeh[1]{
                node[letter, label=#1:{$\id{i}{B1}{0}$}] {H}
            }
            \newcommand\nodee[1]{
                node[letter, fill=\colorblockone, label=#1:{$\coloridone\id{i}{B1}{0}\id{f}{A1}{0}$}] {E}
            }
            \newcommand\nodel[1]{
                node[letter, label=#1:{$\id{i}{B1}{1}$}] {L}
            }
            \newcommand\nodelo[1]{
                node[block, label=#1:{$\id{i}{B1}{1..2}$}] {LO}
            }

            \newcommand\initialstate[3]{
              \path
                #1
                ++#2
                ++(0:0.5)
                ++(#3:0.5) \nodehl{#3};
            }

            \newcommand\inso[3]{
              \path
                #1
                ++#2
                ++(0:0.5)
                ++(#3:0.5) \nodehlo{#3};
            }

            \newcommand\inse[3]{
              \path
                #1
                ++#2
                ++(0:0.5)
                ++(#3:0.5) \nodeh{#3}
                ++(0:\widthletter) \nodee{#3}
                ++(0:\widthletter) \nodel{#3};
            }

            \newcommand\finalstate[3]{
              \path
                #1
                ++#2
                ++(0:0.5)
                ++(#3:0.5) \nodeh{#3}
                ++(0:\widthletter) \nodee{#3}
                ++(0:\widthletter) \nodelo{#3};
            }

            \newcommand\offseta{ (90:0.7) }
            \newcommand\offsetb{ (270:0.7) }

            \path
                node {\textbf{A}}
                ++(0:0.5) node (a) {}
                +(0:16) node (a-end) {}
                +(0:1) node[point] (a-initial) {}
                +(0:5) node (a-ins-e) {}
                +(0:10) node (a-recv-ins-o) {}
                +(0:15) node (a-final) {};

            \initialstate{(a-initial)}{\offseta}{90};
            \draw[dotted] (a) -- (a-initial) (a-final) -- (a-end);

            \path
                ++(270:3) node {\textbf{B}}
                ++(0:0.5) node (b) {}
                +(0:16) node (b-end) {}
                +(0:1) node[point] (b-initial) {}
                +(0:5) node (b-ins-o) {}
                +(0:10) node (b-recv-ins-e) {}
                +(0:15) node (b-final) {};

            \initialstate{(b-initial)}{\offsetb}{-90};
            \draw[dotted] (b) -- (b-initial) (b-final) -- (b-end);

            \only<1> {
                \draw[->, thick] (a-initial) -- (a-final);
            }
            \only<1,2> {
                \draw[->, thick] (b-initial) -- (b-final);
            }
            \only<2,3,4,5> {
                \draw[->, thick] (a-initial) --  (a-ins-e) -- (a-final);
            }
            \onslide<2-> {
                \path (a-ins-e) node[point, label=-170:{$\trm{ins}(H \prec E \prec L)$}] {};
                \inse{(a-ins-e)}{\offseta}{90};
            }
            \only<3,4> {
                \draw[->, thick] (b-initial) --  (b-ins-o) -- (b-final);
            }
            \onslide<3-> {
                \path (b-ins-o) node[point, label=170:{$\trm{ins}(L \prec O)$}] {};
                \inso{(b-ins-o)}{\offsetb}{-90};
            }
            \onslide<4-> {
                \path (a-ins-e) node[label={[xshift=25pt]-10:{$\trm{ins}({\coloridthree\id{i}{B1}{0}\id{f}{A1}{0}},E)$}}] {};
                \draw[->, dashed, shorten >= 1] (a-ins-e) -- (b-recv-ins-e);
            }
            \onslide<5-> {
                \path (b-recv-ins-e) node[point] {};
                \finalstate{(b-recv-ins-e)}{\offsetb}{-90};
                \draw[->, thick] (b-initial) --  (b-ins-o) -- (b-recv-ins-e) -- (b-final);
            }
            \onslide<6-> {
                \path (b-ins-o) node[label={[xshift=25pt]10:{$\trm{ins}(\id{i}{B1}{2},O)$}}] {};
                \path (a-recv-ins-o) node[point] {};
                \draw[->, dashed, shorten >= 1] (b-ins-o) -- (a-recv-ins-o);
                \finalstate{(a-recv-ins-o)}{\offseta}{90};
                \draw[->, thick] (a-initial) --  (a-ins-e) -- (a-recv-ins-o) -- (a-final);
            }
          \end{tikzpicture}
        }
    \end{figure}
\end{frame}

\begin{frame}{Limites de LogootSplit}
    \metroset{block=transparent}
    \begin{block}{Sources de la croissance des métadonnées}
        \begin{itemize}
            \item Augmentation non-bornée de la taille des identifiants
            \item Fragmentation de la séquence en un nombre croissant de blocs
        \end{itemize}
    \end{block}
    \vspace{-1.5em}
    \begin{columns}
        \begin{column}{0.6\textwidth}
            \begin{figure}
                \resizebox{0.9 \columnwidth}{!}{
                    \includegraphics[trim=0cm 0.5cm 0cm 0cm, clip]{img/ls-vs-content-snapshot-sizes-7k5.pdf}
                }
                \caption{Taille du contenu comparée à la taille de la séquence LogootSplit}
            \end{figure}
        \end{column}
        \begin{column}{0.4\textwidth}
            \vspace{-2em}
            \begin{center}
                Diminution des performances du point de vue \alert{mémoire, calculs et bande-passante}
            \end{center}
        \end{column}
    \end{columns}
\end{frame}

\begin{frame}{Mitigation du surcoût des CRDTs pour le type Séquence}
    \metroset{block=transparent}
    \begin{block}{L'approche core-nebula \footcite{zawirski:hal-01248197}, pour Treedoc}
        \begin{itemize}
            \item Ré-assigne des identifiants courts aux éléments, \ie les \emph{renomme}
            \item Transforme les opérations \ins et \rmv concurrentes\dots
            \pause
            \item \dots mais ne supportent pas opérations \ren concurrentes
        \end{itemize}
    \end{block}
    \pause
    \begin{center}
        \alert{Inadaptée aux applications pair-à-pair}
    \end{center}
\end{frame}

\begin{frame}[standout]
    \alert{Proposition}

    Mécanisme de renommage supportant les renommages concurrents
\end{frame}
