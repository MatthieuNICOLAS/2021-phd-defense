\section{État de l'art des CRDTs pour le type Séquence}

\begin{frame}{Type de données \emph{Séquence}}
    TODO: Représenter une séquence, puis une exécution provoquant un conflit
    \begin{itemize}
        \item Représente une collection ordonnée et de taille dynamique d'éléments
        \item Deux opérations de modifications, \ins et \rmv
        \item Modifications ne commutent généralement pas
    \end{itemize}
    Un mécanisme de résolution de conflits est nécessaire pour résoudre les divergences
\end{frame}

\begin{frame}{Conflict-free Replicated Data Types (CRDTs) \footcite{shapiro_2011_crdt}}
    \begin{itemize}
        \item Nouvelles spécifications des types de données
        \item Types de données répliquées
        \begin{itemize}
            \item Permettent modifications sans coordination
            \item Garantissent la convergence forte
        \end{itemize}
    \end{itemize}
    \begin{alertblock}{Convergence forte}
        Ensemble des noeuds ayant intégrés le même ensemble de modifications obtient des états équivalents, sans nécessiter d'actions ou messages supplémentaires
    \end{alertblock}
\end{frame}

\begin{frame}{CRDTs et théorie des treillis \footcite{2002-intro-lattices-order-davey}}
    \begin{itemize}
        \item Reposent sur la théorie des treillis
    \end{itemize}
    TODO: Représenter un sup-demi-treillis
    \begin{itemize}
        \item États ordonnés par relation $\leq$, avec état minimal $\bot$
        \item Modification d'un état génère état supérieur ou égal d'après $\leq$
        \item Existe fonction qui fusionne deux états et retourne leur borne supérieure
    \end{itemize}
\end{frame}

\begin{frame}[standout]
    La fonction de fusion d'états est un mécanisme de résolution de conflits automatique
\end{frame}

\begin{frame}{Caractéristiques d'un CRDT}
    \begin{block}{Ce qui définit un CRDT}
        \begin{itemize}
            \item Sémantique de son mécanisme de résolution de conflits
            \item Modèle de synchronisation utilisé
        \end{itemize}
    \end{block}
    \begin{block}{Se focalise sur}
        \begin{itemize}
            \item Spécification forte des séquences répliquées avec entrelacements \footcite{2021-specification-complexity-collaborative-text-editing-attiya}
            \item Synchronisation par opérations
        \end{itemize}
    \end{block}
\end{frame}

\begin{frame}{LogootSplit \footcite{2013-logootsplit}}
    \begin{itemize}
        \item CRDT pour le type Séquence
        \item Appartient à l'approche à identifiants densément ordonnés
    \end{itemize}
\end{frame}

\begin{frame}{Approche à identifiants densément ordonnés}
    \begin{itemize}
        \item Assigne un identifiant de position à chaque élément de la séquence
        \item Utilise les identifiants des éléments pour les ordonner les uns par rapport aux autres
    \end{itemize}
    \begin{block}{Propriétés des identifiants de position \footcite{2009-treedoc-preguica}}
        \begin{itemize}
            \item Unique
            \item Immuable
            \item Ordonnable par une relation d'ordre strict total $\lid$
            \item Appartenant à un espace dense
        \end{itemize}
    \end{block}
\end{frame}

\begin{frame}{Identifiant LogootSplit}
    \begin{itemize}
        \item Composé d'un ou plusieurs tuples $\id{pos}{nodeId~nodeSeq}{offset}$
        \item TODO: Ajouter animations pour expliciter le rôle de chaque composant
    \end{itemize}
    Exemple :
    \begin{equation*}
        \id{i}{A1}{0}\id{r}{B1}{3}\cdots\id{m}{A2}{0}
    \end{equation*}
\end{frame}

\begin{frame}{???}
    TODO: Reprendre exemple de divergence, mais avec LogootSplit pour montrer convergence
\end{frame}

\begin{frame}{Limites de LogootSplit}
    \begin{itemize}
        \item Croissance non-bornée de la taille des identifiants
        \item Fragmentation en blocs courts
    \end{itemize}
    \metroset{block=transparent}
    \begin{block}{Taille du contenu comparé à la taille de la séquence LogootSplit}
        \begin{columns}
            \begin{column}{0.6\textwidth}
                \begin{figure}
                    \resizebox{\columnwidth}{!}{
                        \includegraphics{img/ls-vs-content-snapshot-sizes-7k5.pdf}
                    }
                \end{figure}
            \end{column}
            \begin{column}{0.4\textwidth}
                1\% contenu, 99\% métadonnées
            \end{column}
        \end{columns}
    \end{block}
\end{frame}

\begin{frame}{Mitigation du surcoût des CRDTs pour le type Séquence}
    \metroset{block=transparent}
    \begin{block}{L'approche core-nebula \footcite{zawirski:hal-01248197}}
        \begin{itemize}
            \item Ré-assigne des identifiants courts aux éléments, \ie les \emph{renomme}
            \item Nécessite de transformer les opérations \ins et \rmv concurrentes\dots
            \item \dots et consensus pour effectuer le renommage
        \end{itemize}
    \end{block}
    \alert{Inadaptée aux systèmes P2P sujets au churn}
\end{frame}

\begin{frame}[standout]
    Pouvons-nous proposer un mécanisme de réduction du surcoût des CRDTs pour le type Séquence adapté aux applications LFS, \ie  sans coordination synchrone ?
\end{frame}

% \begin{frame}{Mitigation du surcoût des CRDTs pour le type Séquence}
%     \metroset{block=transparent}
%     \begin{block}{L'approche LSEQ \footcite{lseq2017}}
%         \begin{itemize}
%             \item Structure de données \emph{sur mesure} pour représenter les identifiants
%             \item Ensemble de fonctions d'allocation des identifiants, chacune adaptée à un comportement d'édition différent
%             \item Permet de ralentir la croissance des identifiants
%             \item
%         \end{itemize}
%     \end{block}
% \end{frame}

% \begin{frame}{Research issue}

%   \begin{itemize}
%     \item \textbf{Evergrowing overhead:} impacts memory, bandwidth and CPU
%   \end{itemize}

%   \begin{columns}
%     \begin{column}{0.45\textwidth}
%       \begin{figure}
%         \centering
%         \includegraphics[width=\textwidth]{img/overhead-size.pdf}
%         \caption{Memory footprint of the data structure}
%       \end{figure}
%     \end{column}
%     \begin{column}{0.55\textwidth}
%       \begin{itemize}
%         \item \textbf{Operation count:} 100k
%         \item \textbf{Size of content:} 100KB
%         \item \textbf{Size of data structure:} 20MB
%       \end{itemize}
%     \end{column}
%   \end{columns}

%   \centering
%   \alert{How to reduce the overhead introduced by the data structure?}
% \end{frame}
