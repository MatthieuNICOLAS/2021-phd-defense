\metroset{sectionpage=progressbar}
\section{Conclusion générale \& Perspectives}
\metroset{sectionpage=none}

\begin{frame}{Conclusion}
  \metroset{block=transparent}
  \begin{block}{Contributions}
    \begin{itemize}
      \item Conception d'un mécanisme de renommage pour CRDTs pour le type Séquence à identifiants densément ordonnés
      \begin{itemize}
        \item Implémentation et instrumentation de RenamableLogootSplit et de ses dépendances (protocole d'appartenance au réseau, couche de livraison)
      \end{itemize}
      \pause
      \item Comparaison des différents modèles de synchronisation pour CRDTs\dots
      \item \dots et des différentes approches pour CRDTs pour le type Séquence
    \end{itemize}
  \end{block}
\end{frame}

\begin{frame}{Limites \& perspectives}
  \metroset{block=transparent}
  \begin{block}{Limites de RenamableLogootSplit}
    \begin{itemize}
      \item Surcoût fonction du nombre d'opérations \ren concurrentes
      \item Stabilité causale requise pour supprimer les métadonnées
    \end{itemize}
  \end{block}
  \begin{block}{Perspectives autour de RenamableLogootSplit}
    \begin{itemize}
      \item Comment définir des relations \emph{priority} $\lepoch$ qui minimisent les renommages vains ?
      \item Peut-on prouver formellement la correction RenamableLogootSplit ?
    \end{itemize}
  \end{block}
  \pause
  \begin{block}{Perspectives autour des CRDTs}
    \begin{itemize}
      \item Doit-on encore concevoir CRDTs synchronisés par états ou opérations ?
      \item Peut-on proposer un framework pour conception de CRDTs synchronisés par opérations ?
    \end{itemize}
  \end{block}
\end{frame}

\begin{frame}[standout]
  Merci de votre attention, avez-vous des questions ?
  \vspace{3em}
  \begin{center}
    \ccby
  \end{center}
\end{frame}
