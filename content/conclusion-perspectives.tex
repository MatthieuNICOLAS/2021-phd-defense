\metroset{sectionpage=progressbar}
\section{Conclusion générale \& Perspectives}
\metroset{sectionpage=none}

\begin{frame}{Conclusion}
  \metroset{block=transparent}
  \begin{block}{Contributions}
    \begin{itemize}
      \item Mécanisme de renommage pour CRDTs pour le type Séquence à identifiants densément ordonnés
      \item Implémentation de RenamableLogootSplit et de ses dépendances (protocole d'appartenance au réseau, couche de livraison) dans MUTE
      \item Comparaison des différents modèles de synchronisation pour CRDTs, ainsi que
      \item Comparaison des différentes approches pour CRDTs pour le type Séquence
    \end{itemize}
  \end{block}
\end{frame}

\begin{frame}{Perspectives}
  \metroset{block=transparent}
  \begin{block}{RenamableLogootSplit}
    \begin{itemize}
      \item Peut-on définir une relation \emph{priority} $\lepoch$ réduisant calculs à échelle du système ?
    \end{itemize}
  \end{block}
  \pause
  \begin{block}{CRDTs}
    \begin{itemize}
      \item Doit-on encore concevoir CRDTs synchronisés par états ou opérations ?
      \item Peut-on proposer un framework pour conception de CRDTs synchronisés par opérations ?
    \end{itemize}
  \end{block}
\end{frame}

\begin{frame}{Ouverture : Doit-on encore concevoir CRDTs synchronisés par états ou opérations ?}
  \begin{table}[!ht]
    \resizebox{\columnwidth}{!}{
      \begin{tabular}{lccc}
        \toprule
                                                  & Sync. par états & Sync. par opérations    & Sync. par diff. d'états \\
        \midrule
        Forme un sup-demi-treillis                  & \checkmark  & \checkmark  & \checkmark  \\
        Intègre modifications par fusion d'états  & \checkmark  & \ballotx    & \checkmark  \\
        Intègre modifications par élts irréductibles & \ballotx    & \checkmark  & \checkmark  \\
        Résiste nativ. aux défaillances réseau           & \checkmark  & \ballotx    & \checkmark  \\
        Adapté pour systèmes temps réel           & \ballotx    & \checkmark  & \checkmark  \\
        Offre nativ. modèle de cohérence causale             & \checkmark  & \ballotx    & \ballotx  \\
        \bottomrule
      \end{tabular}
    }
  \end{table}
  \begin{itemize}
    \item Synchronisation par différences offre meilleur des mondes\dots
    \item \dots y a-t-il encore un intérêt aux autres modèles, \eg pour composition ou sécurité ?
  \end{itemize}
\end{frame}

\begin{frame}[standout]
  Merci de votre attention, avez-vous des questions ?
  \vspace{3em}
  \begin{center}
    \ccby
  \end{center}
\end{frame}
