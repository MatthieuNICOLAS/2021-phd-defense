\subsection{Intégration des opérations \ins et \rmv concurrentes}

\begin{frame}[fragile]{Intégration des opérations \ins et \rmv concurrentes}
  \begin{figure}
    \resizebox{\textwidth}{!}{
      \begin{tikzpicture}
        \newcommand\nodeh[1]{
            node[letter, label=#1:{$\id{i}{B1}{0}$}] {H}
        }
        \newcommand\nodee[1]{
            node[letter, fill=\colorblockone, label=#1:{$\coloridone\id{i}{B1}{0}\id{f}{A1}{0}$}] {E}
        }
        \newcommand\nodelo[1]{
            node[block, label=#1:{$\id{i}{B1}{1..2}$}] {LO}
        }
        \newcommand\renhelo[1]{
            node[block, fill=\colorblocktwo, label=#1:{$\coloridtwo\id{i}{A2}{0..3}$}] {HELO}
        }
        \newcommand\nodel[1]{
            node[letter, fill=\colorblockthree, label=#1:{$\coloridthree\id{i}{B1}{0}\id{m}{B2}{0}$}] {L}
        }
        \newcommand\crossl[1]{
            node[letter, cross, fill=\colorblockthree, label=#1:{$\coloridthree\id{i}{B1}{0}\id{m}{B2}{0}$}] {L}
        }

        \newcommand\initialstate[3]{
            \path
                #1
                ++#2
                ++(0:0.5)
                ++(#3:0.5) \nodeh{-#3}
                ++(0:\widthletter) \nodee{#3}
                ++(0:\widthletter) \nodelo{-#3};
        }

        \newcommand\statehelo[3]{
            \path
                #1
                ++#2
                ++(0:0.5)
                ++(#3:0.5) \renhelo{-#3};
        }

        \newcommand\insl[3]{
            \path
            #1
            ++#2
            ++(0:0.5)
            ++(#3:0.5) \nodeh{-#3}
            ++(0:\widthletter) \nodee{#3}
            ++(0:\widthletter) \nodel{-#3}
            ++(0:\widthletter) \nodelo{#3};
        }

        \newcommand\finalstate[3]{
            \path
                #1
                ++#2
                ++(0:0.5)
                ++(#3:0.5) \renhelo{-#3}
                ++(0:\widthblock) \crossl{#3};
        }

        \newcommand\offseta{ (90:0.7) }
        \newcommand\offsetb{ (-90:0.7) }

        \path
            node {\textbf{A}}
            ++(0:0.5) node (a) {}
            +(0:18) node (a-end) {}
            +(0:1) node[point] (a-initial) {}
            +(0:7) node[point, label=-170:{$\trm{ren}()$}, label={[xshift=0pt]-10:{$\trm{ren}(A,2)$}}] (a-ren-a1) {}
            +(0:13) node[point] (a-recv-ins-l) {}
            +(0:17) node (a-final) {};

        \initialstate{(a-initial)}{\offseta}{90};
        \statehelo{(a-ren-a1)}{\offseta}{90};
        \finalstate{(a-recv-ins-l)}{\offseta}{90};

        \draw[dotted] (a) -- (a-initial) (a-final) -- (a-end);
        \draw[->, thick] (a-initial) --  (a-ren-a1) -- (a-recv-ins-l) -- (a-final);

        \path
            ++(270:3) node {\textbf{B}}
            ++(0:0.5) node (b) {}
            +(0:18) node (b-end) {}
            +(0:1) node[point] (b-initial) {}
            +(0:7) node[point, label=170:{$\trm{ins}(E \prec L \prec O)$}, label={[xshift=45pt]10:{$\trm{ins}({\coloridthree\id{i}{B1}{0}\id{m}{B2}{0}},L)$}}] (b-ins-l) {}
            +(0:17) node (b-final) {};


        \initialstate{(b-initial)}{\offsetb}{-90};
        \insl{(b-ins-l)}{\offsetb}{-90};

        \draw[dotted] (b) -- (b-initial) (b-final) -- (b-end);
        \draw[->, thick] (b-initial) --  (b-ins-l) -- (b-final);

        \draw[->, dashed, shorten >= 1] (b-ins-l) -- (a-recv-ins-l);
      \end{tikzpicture}
    }
  \end{figure}
  \begin{itemize}
    \item Peuvent générer opérations concurrentes aux opérations \ren
    \item Produisent anomalies si intégrées telles qu'elles
  \end{itemize}
\end{frame}

\begin{frame}[fragile]{Correction de l'intégration des opérations concurrentes}
  \begin{figure}
    \resizebox{0.7 \textwidth}{!}{
      \begin{tikzpicture}
        \newcommand\nodeh[1]{
            node[letter, label=#1:{$\id{i}{B1}{0}$}] {H}
        }
        \newcommand\nodee[1]{
            node[letter, fill=\colorblockone, label=#1:{\coloridone$\id{i}{B1}{0}\id{f}{A1}{0}$}] {E}
        }
        \newcommand\nodelo[1]{
            node[block, label=#1:{$\id{i}{B1}{1..2}$}] {LO}
        }
        \newcommand\renhelo[1]{
            node[block, fill=\colorblocktwo, label=#1:{\coloridtwo$\id{i}{A2}{0..3}$}] {HELO}
        }
        \newcommand\nodel[1]{
            node[letter, fill=\colorblockthree,label=#1:{\coloridthree$\id{i}{B1}{0}\id{m}{B2}{0}$}] {L}
        }
        \newcommand\renhe[1]{
            node[block, fill=\colorblocktwo, label=#1:{\coloridtwo$\id{i}{A2}{0..1}$}] {HE}
        }
        \newcommand\renl[1]{
            node[letter, fill=\colorblockfour, label=#1:{\coloridfour$\id{i}{A2}{1}\id{i}{B1}{0}\id{m}{B2}{0}$}] {L}
        }
        \newcommand\renlo[1]{
            node[block, fill=\colorblocktwo, label=#1:{\coloridtwo$\id{i}{A2}{2..3}$}] {LO}
        }

        \newcommand\initialstate[3]{
            \path
                #1
                ++#2
                ++(0:0.5)
                ++(#3:0.5) node[epoch] {$\epoch{0}$}
                ++(0:1.25 * \widthoriginepoch) \nodeh{-#3}
                ++(0:\widthletter) \nodee{#3}
                ++(0:\widthletter) \nodelo{-#3};
        }

        \newcommand\statehelo[3]{
            \path
                #1
                ++#2
                ++(0:0.5)
                ++(#3:0.5) node[epoch] {$\epoch{A2}$}
                ++(0:1.18 * \widthepoch) \renhelo{-#3};
        }

        \newcommand\insl[3]{
            \path
            #1
            ++#2
            ++(0:0.5)
            ++(#3:0.5) node[epoch] {$\epoch{0}$}
            ++(0:1.25 * \widthoriginepoch) \nodeh{-#3}
            ++(0:\widthletter) \nodee{#3}
            ++(0:\widthletter) \nodel{-#3}
            ++(0:\widthletter) \nodelo{#3};
        }

        \newcommand\finalstate[3]{
            \path
                #1
                ++#2
                ++(0:0.5)
                ++(#3:0.5) node[epoch] {$\epoch{A2}$}
                ++(0:1.18 * \widthepoch) \renhe{-#3}
                ++(0:\widthblock) \renl{#3}
                ++(0:\widthletter) \renlo{-#3};
        }

        \newcommand\offseta{ (90:0.7) }
        \newcommand\offsetb{ (-90:0.7) }

        \path
            node {\textbf{A}}
            ++(0:0.5) node (a) {}
            +(0:18) node (a-end) {}
            +(0:1) node[point] (a-initial) {}
            +(0:7) node[point, label=-170:{$\trm{ren}()$}, label={[xshift=0pt]-10:{$\trm{ren}(\epoch{0}, A,2)$}}] (a-ren-a1) {}
            +(0:13) node[point] (a-recv-ins-l) {}
            +(0:17) node (a-final) {};

        \initialstate{(a-initial)}{\offseta}{90};
        \statehelo{(a-ren-a1)}{\offseta}{90};
        \finalstate{(a-recv-ins-l)}{\offseta}{90};

        \draw[dotted] (a) -- (a-initial) (a-final) -- (a-end);
        \draw[->, thick] (a-initial) --  (a-ren-a1) -- (a-recv-ins-l) -- (a-final);

        \path
            ++(270:3) node {\textbf{B}}
            ++(0:0.5) node (b) {}
            +(0:18) node (b-end) {}
            +(0:1) node[point] (b-initial) {}
            +(0:7) node[point, label=170:{$\trm{ins}(E \prec L \prec O)$}, label={[xshift=45pt]10:{$\trm{ins}(\epoch{0}, {\coloridthree\id{i}{B1}{0}\id{m}{B2}{0}},L)$}}] (b-ins-l) {}
            +(0:17) node (b-final) {};


        \initialstate{(b-initial)}{\offsetb}{-90};
        \insl{(b-ins-l)}{\offsetb}{-90};

        \draw[dotted] (b) -- (b-initial) (b-final) -- (b-end);
        \draw[->, thick] (b-initial) --  (b-ins-l) -- (b-final);

        \draw[->, dashed, shorten >= 1] (b-ins-l) -- (a-recv-ins-l);
      \end{tikzpicture}
    }
  \end{figure}
  \begin{itemize}
    \item Ajout d'un système d'époque pour identifier les opérations concurrentes à une opération \ren
    \item Transformation avant intégration des opérations concurrentes
  \end{itemize}
  \begin{block}{Exemple avec {$\coloridthree\id{i}{B1}{0}\id{m}{B2}{0}$}}
    \begin{itemize}
      \item Trouver son prédecesseur à l'époque d'origine $\epoch{0}$ : $\coloridone\id{i}{B1}{0}\id{f}{A1}{0}$
      \item Trouver son équivalent à l'époque destination $\epoch{A1}$ : $\coloridtwo\id{i}{A2}{1}$
      \item Concaténer ce dernier à l'identifiant pour obtenir son équivalent à $\epoch{A1}$ : $\coloridfour\id{i}{A2}{1}\id{i}{B1}{0}\id{m}{B2}{0}$
    \end{itemize}
  \end{block}
\end{frame}
