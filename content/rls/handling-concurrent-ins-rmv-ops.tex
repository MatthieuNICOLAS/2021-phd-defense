\subsection{Intégration des opérations \ins et \rmv concurrentes}

\begin{frame}[fragile]{Interactions avec opérations \ins et \rmv concurrentes}
  \begin{figure}
    \resizebox{\textwidth}{!}{
      \begin{tikzpicture}
        \newcommand\nodeh[1]{
            node[letter, label=#1:{$\id{i}{B1}{0}$}] {H}
        }
        \newcommand\nodee[1]{
            node[letter, fill=\colorblockone, label=#1:{$\coloridone\id{i}{B1}{0}\id{f}{A1}{0}$}] {E}
        }
        \newcommand\nodelo[1]{
            node[block, label=#1:{$\id{i}{B1}{1..2}$}] {LO}
        }
        \newcommand\renhelo[1]{
            node[block, fill=\colorblocktwo, label=#1:{$\coloridtwo\id{i}{A2}{0..3}$}] {HELO}
        }
        \newcommand\nodel[1]{
            node[letter, fill=\colorblockthree, label=#1:{$\coloridthree\id{i}{B1}{0}\id{m}{B2}{0}$}] {L}
        }
        \newcommand\crossl[1]{
            node[letter, cross, fill=\colorblockthree, label=#1:{$\coloridthree\id{i}{B1}{0}\id{m}{B2}{0}$}] {L}
        }

        \newcommand\initialstate[3]{
            \path
                #1
                ++#2
                ++(0:0.5)
                ++(#3:0.5) \nodeh{-#3}
                ++(0:\widthletter) \nodee{#3}
                ++(0:\widthletter) \nodelo{-#3};
        }

        \newcommand\statehelo[3]{
            \path
                #1
                ++#2
                ++(0:0.5)
                ++(#3:0.5) \renhelo{-#3};
        }

        \newcommand\insl[3]{
            \path
            #1
            ++#2
            ++(0:0.5)
            ++(#3:0.5) \nodeh{-#3}
            ++(0:\widthletter) \nodee{#3}
            ++(0:\widthletter) \nodel{-#3}
            ++(0:\widthletter) \nodelo{#3};
        }

        \newcommand\finalstate[3]{
            \path
                #1
                ++#2
                ++(0:0.5)
                ++(#3:0.5) \renhelo{-#3}
                ++(0:\widthblock) \nodel{#3};
        }

        \newcommand\finalstatecrossed[3]{
            \path
                #1
                ++#2
                ++(0:0.5)
                ++(#3:0.5) \renhelo{-#3}
                ++(0:\widthblock) \crossl{#3};
        }

        \newcommand\offseta{ (90:0.7) }
        \newcommand\offsetb{ (-90:0.7) }

        \path
            node {\textbf{A}}
            ++(0:0.5) node (a) {}
            +(0:16) node (a-end) {}
            +(0:1) node[point] (a-initial) {}
            +(0:6) node[point, label=-170:{$\trm{ren}()$}, label={[xshift=0pt]-10:{$\trm{ren}(A,2,\trm{renIds})$}}] (a-ren-a1) {}
            +(0:12) node (a-recv-ins-l) {}
            +(0:15) node (a-final) {};
        \path<3->
            (a-recv-ins-l) node[point] {};

        \initialstate{(a-initial)}{\offseta}{90};
        \statehelo{(a-ren-a1)}{\offseta}{90};
        \only<4>{\finalstate{(a-recv-ins-l)}{\offseta}{90};}
        \onslide<5->{\finalstatecrossed{(a-recv-ins-l)}{\offseta}{90};}

        \draw[dotted] (a) -- (a-initial) (a-final) -- (a-end);
        \draw<-2>[->, thick] (a-initial) --  (a-ren-a1) -- (a-final);
        \draw<3->[->, thick] (a-initial) --  (a-ren-a1) -- (a-recv-ins-l) -- (a-final);

        \path
            ++(270:2) node {\textbf{B}}
            ++(0:0.5) node (b) {}
            +(0:16) node (b-end) {}
            +(0:1) node[point] (b-initial) {}
            +(0:6) node (b-ins-l) {}
            +(0:15) node (b-final) {};
        \path<2->
            (b-ins-l) node[point, label=170:{$\trm{ins}(E \prec L \prec O)$}] {};
        \path<3->
            (b-ins-l) node[label={[xshift=45pt]10:{$\trm{ins}({\coloridthree\id{i}{B1}{0}\id{m}{B2}{0}},L)$}}] {};

        \initialstate{(b-initial)}{\offsetb}{-90};
        \onslide<2->{\insl{(b-ins-l)}{\offsetb}{-90};}

        \draw[dotted] (b) -- (b-initial) (b-final) -- (b-end);
        \draw<1>[->, thick] (b-initial) -- (b-final);
        \draw<2->[->, thick] (b-initial) --  (b-ins-l) -- (b-final);

        \draw<3->[->, dashed, shorten >= 1] (b-ins-l) -- (a-recv-ins-l);
      \end{tikzpicture}
    }
  \end{figure}
  \begin{itemize}
    \item Peuvent générer opérations concurrentes aux opérations \ren
    \item<5-> Produisent anomalies si intégrées naïvement
  \end{itemize}
  \onslide<6>{
    \begin{center}
      \alert{Nécessité d'un mécanisme dédié}
    \end{center}
  }
\end{frame}

\begin{frame}{Mécanisme de résolution de conflits}
    \begin{block}{Besoins}
        \begin{enumerate}
            \item
                \only<1>{Détecter les opérations concurrentes aux opérations \ren}
                \only<2>{\alert{Détecter les opérations concurrentes aux opérations \ren}}
            \item Prendre en compte effet des opérations \ren lors de l'intégration des opérations concurrentes
        \end{enumerate}
    \end{block}
\end{frame}

\begin{frame}[fragile]{Détection des opérations concurrentes à opération \ren}
    % \begin{center}
    %     \alert{Besoin de détecter opérations concurrentes à opération \ren}
    % \end{center}
    \vspace{-1.5em}
    \begin{figure}
        \resizebox{\textwidth}{!}{
          \begin{tikzpicture}
            \newcommand\nodeh[1]{
                node[letter, label=#1:{$\id{i}{B1}{0}$}] {H}
            }
            \newcommand\nodee[1]{
                node[letter, fill=\colorblockone, label=#1:{\coloridone$\id{i}{B1}{0}\id{f}{A1}{0}$}] {E}
            }
            \newcommand\nodelo[1]{
                node[block, label=#1:{$\id{i}{B1}{1..2}$}] {LO}
            }
            \newcommand\renhelo[1]{
                node[block, fill=\colorblocktwo, label=#1:{\coloridtwo$\id{i}{A2}{0..3}$}] {HELO}
            }
            \newcommand\nodel[1]{
                node[letter, fill=\colorblockthree,label=#1:{\coloridthree$\id{i}{B1}{0}\id{m}{B2}{0}$}] {L}
            }
            \newcommand\renhe[1]{
                node[block, fill=\colorblocktwo, label=#1:{\coloridtwo$\id{i}{A2}{0..1}$}] {HE}
            }
            \newcommand\renl[1]{
                node[letter, fill=\colorblockfour, label=#1:{\coloridfour$\id{i}{A2}{1}\id{i}{B1}{0}\id{m}{B2}{0}$}] {L}
            }
            \newcommand\renlo[1]{
                node[block, fill=\colorblocktwo, label=#1:{\coloridtwo$\id{i}{A2}{2..3}$}] {LO}
            }

            \newcommand\initialstate[3]{
                \path
                    #1
                    ++#2
                    ++(0:0.5)
                    ++(#3:0.5)
                    ++(0:1.25 * \widthoriginepoch) \nodeh{-#3}
                    ++(0:\widthletter) \nodee{#3}
                    ++(0:\widthletter) \nodelo{-#3};
            }

            \newcommand\initialstatewithepoch[3]{
                \path
                    #1
                    ++#2
                    ++(0:0.5)
                    ++(#3:0.5) node[epoch] {$\epoch{0}$}
                    ++(0:1.25 * \widthoriginepoch) \nodeh{-#3}
                    ++(0:\widthletter) \nodee{#3}
                    ++(0:\widthletter) \nodelo{-#3};
            }

            \newcommand\statehelo[3]{
                \path
                    #1
                    ++#2
                    ++(0:0.5)
                    ++(#3:0.5)
                    ++(0:1.18 * \widthepoch) \renhelo{-#3};
            }

            \newcommand\statehelowithepoch[3]{
                \path
                    #1
                    ++#2
                    ++(0:0.5)
                    ++(#3:0.5) node[epoch] {$\epoch{A2}$}
                    ++(0:1.18 * \widthepoch) \renhelo{-#3};
            }

            \newcommand\insl[3]{
                \path
                #1
                ++#2
                ++(0:0.5)
                ++(#3:0.5)
                ++(0:1.25 * \widthoriginepoch) \nodeh{-#3}
                ++(0:\widthletter) \nodee{#3}
                ++(0:\widthletter) \nodel{-#3}
                ++(0:\widthletter) \nodelo{#3};
            }

            \newcommand\inslwithepoch[3]{
                \path
                #1
                ++#2
                ++(0:0.5)
                ++(#3:0.5) node[epoch] {$\epoch{0}$}
                ++(0:1.25 * \widthoriginepoch) \nodeh{-#3}
                ++(0:\widthletter) \nodee{#3}
                ++(0:\widthletter) \nodel{-#3}
                ++(0:\widthletter) \nodelo{#3};
            }

            \newcommand\offseta{ (90:0.7) }
            \newcommand\offsetb{ (-90:0.7) }

            \path
                node {\textbf{A}}
                ++(0:0.5) node (a) {}
                +(0:15) node (a-end) {}
                +(0:1) node[point] (a-initial) {}
                +(0:7) node[point, label=-170:{$\trm{ren}()$}, ] (a-ren-a1) {}
                +(0:13) node[point] (a-recv-ins-l) {}
                +(0:14) node (a-final) {};
            \path<5>
                (a-ren-a1) node[label={[xshift=0pt]-10:{$\trm{ren}(\epoch{0}, A,2,\trm{renIds}_{A2})$}}] {};

            \onslide<1-2>{\initialstate{(a-initial)}{\offseta}{90};}
            \onslide<1-3>{\statehelo{(a-ren-a1)}{\offseta}{90};}
            \onslide<3->{\initialstatewithepoch{(a-initial)}{\offseta}{90};}
            \onslide<4->{\statehelowithepoch{(a-ren-a1)}{\offseta}{90};}

            \draw[dotted] (a) -- (a-initial) (a-final) -- (a-end);
            \draw[->, thick] (a-initial) --  (a-ren-a1) -- (a-recv-ins-l) -- (a-final);

            \path
                ++(270:2) node {\textbf{B}}
                ++(0:0.5) node (b) {}
                +(0:15) node (b-end) {}
                +(0:1) node[point] (b-initial) {}
                +(0:7) node[point, label=170:{$\trm{ins}(E \prec L \prec O)$}] (b-ins-l) {}
                +(0:14) node (b-final) {};
            \path<5>
                (b-ins-l) node[label={[xshift=45pt]10:{$\trm{ins}(\epoch{0}, {\coloridthree\id{i}{B1}{0}\id{m}{B2}{0}},L)$}}] {};


            \onslide<1-2>{
                \initialstate{(b-initial)}{\offsetb}{-90};
                \insl{(b-ins-l)}{\offsetb}{-90};
            }
            \onslide<3->{
                \initialstatewithepoch{(b-initial)}{\offsetb}{-90};
                \inslwithepoch{(b-ins-l)}{\offsetb}{-90};
            }

            \draw[dotted] (b) -- (b-initial) (b-final) -- (b-end);
            \draw[->, thick] (b-initial) --  (b-ins-l) -- (b-final);

            \draw[->, dashed, shorten >= 1] (b-ins-l) -- (a-recv-ins-l);
          \end{tikzpicture}
        }
    \end{figure}
    \metroset{block=transparent}
    \begin{block}<2->{Ajout mécanisme d'époques}
        \begin{itemize}
            \item<3-> Séquence commence à époque d'origine, notée $\epoch{0}$
            \item<4-> \ren font progresser à nouvelle époque, $\epoch{\trm{nodeId}~\trm{nodeSeq}}$
            \item<5-> Opérations labellisées avec époque de génération
        \end{itemize}
    \end{block}
\end{frame}

\begin{frame}{Mécanisme de résolution de conflits}
    \begin{block}{Besoins}
        \begin{enumerate}
            \item {\color{gray}Détecter les opérations concurrentes aux opérations \ren}
            \item
                \only<1>{{Prendre en compte effet des opérations \ren lors de l'intégration des opérations concurrentes}}
                \only<2>{\alert{Prendre en compte effet des opérations \ren lors de l'intégration des opérations concurrentes}}
        \end{enumerate}
    \end{block}
\end{frame}

\begin{frame}[fragile]{Intégration des opérations \ins et \rmv concurrentes}
  \vspace{-1.5em}
  \begin{figure}
    \resizebox{\textwidth}{!}{
      \begin{tikzpicture}
        \newcommand\nodeh[1]{
            node[letter, label=#1:{$\id{i}{B1}{0}$}] {H}
        }
        \newcommand\nodee[1]{
            node[letter, fill=\colorblockone, label=#1:{\coloridone$\id{i}{B1}{0}\id{f}{A1}{0}$}] {E}
        }
        \newcommand\nodelo[1]{
            node[block, label=#1:{$\id{i}{B1}{1..2}$}] {LO}
        }
        \newcommand\renhelo[1]{
            node[block, fill=\colorblocktwo, label=#1:{\coloridtwo$\id{i}{A2}{0..3}$}] {HELO}
        }
        \newcommand\nodel[1]{
            node[letter, fill=\colorblockthree,label=#1:{\coloridthree$\id{i}{B1}{0}\id{m}{B2}{0}$}] {L}
        }
        \newcommand\renhe[1]{
            node[block, fill=\colorblocktwo, label=#1:{\coloridtwo$\id{i}{A2}{0..1}$}] {HE}
        }
        \newcommand\renl[1]{
            node[letter, fill=\colorblockfour, label=#1:{\coloridfour$\id{i}{A2}{1}\id{i}{B1}{0}\id{m}{B2}{0}$}] {L}
        }
        \newcommand\renlo[1]{
            node[block, fill=\colorblocktwo, label=#1:{\coloridtwo$\id{i}{A2}{2..3}$}] {LO}
        }

        \newcommand\statehelo[3]{
            \path
                #1
                ++#2
                ++(0:0.5)
                ++(#3:0.5) node[epoch] {$\epoch{A2}$}
                ++(0:1.18 * \widthepoch) \renhelo{-#3};
        }

        \newcommand\insl[3]{
            \path
            #1
            ++#2
            ++(0:0.5)
            ++(#3:0.5) node[epoch] {$\epoch{0}$}
            ++(0:1.25 * \widthoriginepoch) \nodeh{-#3}
            ++(0:\widthletter) \nodee{#3}
            ++(0:\widthletter) \nodel{-#3}
            ++(0:\widthletter) \nodelo{#3};
        }

        \newcommand\statehellowithoutid[3]{
            \path
            #1
            ++#2
            ++(0:0.5)
            ++(#3:0.5) node[epoch] {$\epoch{A2}$}
            ++(0:1.18 * \widthepoch) \renhe{-#3}
            ++(0:\widthblock) node[letter, fill=\colorblockfour, label=#3:{\coloridfour ?}] {L}
            ++(0:\widthletter) \renlo{-#3};
        }

        \newcommand\finalstate[3]{
            \path
                #1
                ++#2
                ++(0:0.5)
                ++(#3:0.5) node[epoch] {$\epoch{A2}$}
                ++(0:1.18 * \widthepoch) \renhe{-#3}
                ++(0:\widthblock) \renl{#3}
                ++(0:\widthletter) \renlo{-#3};
        }

        \newcommand\offseta{ (90:0.7) }
        \newcommand\offsetb{ (-90:0.7) }

        \path
            node {\textbf{A}}
            ++(0:0.5) node (a) {}
            +(0:13) node (a-end) {}
            +(0:1) node[point] (a-initial) {}
            +(0:7) node[point] (a-recv-ins-l) {}
            +(0:12) node (a-final) {};

        \statehelo{(a-initial)}{\offseta}{90};
        \statehellowithoutid{(a-recv-ins-l)}{\offseta}{90};
        %\finalstate{(a-recv-ins-l)}{\offseta}{90};

        \draw[dotted] (a) -- (a-initial) (a-final) -- (a-end);
        \draw[->, thick] (a-initial) -- (a-recv-ins-l) -- (a-final);

        \path
            ++(270:2) node {\textbf{B}}
            ++(0:0.5) node (b) {}
            +(0:13) node (b-end) {}
            +(0:1) node[point, label={[xshift=45pt]10:{$\trm{ins}(\epoch{0}, {\coloridthree\id{i}{B1}{0}\id{m}{B2}{0}},L)$}}] (b-initial) {}
            +(0:12) node (b-final) {};

        \insl{(b-initial)}{\offsetb}{-90};

        \draw[dotted] (b) -- (b-initial) (b-final) -- (b-end);
        \draw[->, thick] (b-initial) -- (b-final);

        \draw[->, dashed, shorten >= 1] (b-initial) -- (a-recv-ins-l);
      \end{tikzpicture}
    }
  \end{figure}
  \metroset{block=transparent}
  \begin{block}<2->{Ajout d'un mécanisme de transformation des opérations \ins et \rmv concurrentes}
    \begin{itemize}
        \item<3-> Prend la forme de l'algorithme \texttt{renameId}
        \item<4> \alert{Inclure l'effet de l'opération \ren dans l'opération transformée}
    \end{itemize}
  \end{block}
\end{frame}

\begin{frame}[fragile]{Exemple de \texttt{renameId}}
    \vspace{-0.8 em}
    \begin{figure}
      \resizebox{0.9 \textwidth}{!}{
        \begin{tikzpicture}
          \newcommand\nodeh[1]{
              node[letter, label=#1:{$\id{i}{B1}{0}$}] {H}
          }
          \newcommand\nodee[1]{
              node[letter, fill=\colorblockone, label=#1:{\coloridone$\id{i}{B1}{0}\id{f}{A1}{0}$}] {E}
          }
          \newcommand\nodelo[1]{
              node[block, label=#1:{$\id{i}{B1}{1..2}$}] {LO}
          }
          \newcommand\renhelo[1]{
              node[block, fill=\colorblocktwo, label=#1:{\coloridtwo$\id{i}{A2}{0..3}$}] {HELO}
          }
          \newcommand\nodel[1]{
              node[letter, fill=\colorblockthree,label=#1:{\coloridthree$\id{i}{B1}{0}\id{m}{B2}{0}$}] {L}
          }
          \newcommand\renhe[1]{
              node[block, fill=\colorblocktwo, label=#1:{\coloridtwo$\id{i}{A2}{0..1}$}] {HE}
          }
          \newcommand\renl[1]{
              node[letter, fill=\colorblockfour, label=#1:{\coloridfour$\id{i}{A2}{1}\id{i}{B1}{0}\id{m}{B2}{0}$}] {L}
          }
          \newcommand\renlo[1]{
              node[block, fill=\colorblocktwo, label=#1:{\coloridtwo$\id{i}{A2}{2..3}$}] {LO}
          }

          \newcommand\statehelo[3]{
              \path
                  #1
                  ++#2
                  ++(0:0.5)
                  ++(#3:0.5) node[epoch] {$\epoch{A2}$}
                  ++(0:1.18 * \widthepoch) \renhelo{-#3};
          }

          \newcommand\insl[3]{
              \path
              #1
              ++#2
              ++(0:0.5)
              ++(#3:0.5) node[epoch] {$\epoch{0}$}
              ++(0:1.25 * \widthoriginepoch) \nodeh{-#3}
              ++(0:\widthletter) \nodee{#3}
              ++(0:\widthletter) \nodel{-#3}
              ++(0:\widthletter) \nodelo{#3};
          }

          \newcommand\statehellowithoutid[3]{
              \path
              #1
              ++#2
              ++(0:0.5)
              ++(#3:0.5) node[epoch] {$\epoch{A2}$}
              ++(0:1.18 * \widthepoch) \renhe{-#3}
              ++(0:\widthblock) node[letter, fill=\colorblockfour, label=#3:{\coloridfour ?}] {L}
              ++(0:\widthletter) \renlo{-#3};
          }

          \newcommand\finalstate[3]{
              \path
                  #1
                  ++#2
                  ++(0:0.5)
                  ++(#3:0.5) node[epoch] {$\epoch{A2}$}
                  ++(0:1.18 * \widthepoch) \renhe{-#3}
                  ++(0:\widthblock) \renl{#3}
                  ++(0:\widthletter) \renlo{-#3};
          }

          \newcommand\offseta{ (90:0.7) }
          \newcommand\offsetb{ (-90:0.7) }

          \path
              node {\textbf{A}}
              ++(0:0.5) node (a) {}
              +(0:13) node (a-end) {}
              +(0:1) node[point] (a-initial) {}
              +(0:7) node[point] (a-recv-ins-l) {}
              +(0:12) node (a-final) {};

          \statehelo{(a-initial)}{\offseta}{90};
          \onslide<1-4>{
            \statehellowithoutid{(a-recv-ins-l)}{\offseta}{90};
          }
          \onslide<5>{
            \finalstate{(a-recv-ins-l)}{\offseta}{90};
          }

          \draw[dotted] (a) -- (a-initial) (a-final) -- (a-end);
          \draw[->, thick] (a-initial) -- (a-recv-ins-l) -- (a-final);

          \path
              ++(270:2) node {\textbf{B}}
              ++(0:0.5) node (b) {}
              +(0:13) node (b-end) {}
              +(0:1) node[point, label={[xshift=45pt]10:{$\trm{ins}(\epoch{0}, {\coloridthree\id{i}{B1}{0}\id{m}{B2}{0}},L)$}}] (b-initial) {}
              +(0:12) node (b-final) {};

          \insl{(b-initial)}{\offsetb}{-90};

          \draw[dotted] (b) -- (b-initial) (b-final) -- (b-end);
          \draw[->, thick] (b-initial) -- (b-final);

          \draw[->, dashed, shorten >= 1] (b-initial) -- (a-recv-ins-l);
        \end{tikzpicture}
      }
    \end{figure}
    \vspace{-1em}
    \textbf{Rappel : }
    \begin{equation*}
        \trm{renIds}_{A2} = [\id{i}{B1}{0},\id{i}{B1}{0}\id{f}{A1}{0},\id{i}{B1}{1},\id{i}{B1}{2}]
    \end{equation*}
    \vspace{-1em}
    \begin{block}{Exemple avec {$\coloridthree\id{i}{B1}{0}\id{m}{B2}{0}$}}
      \begin{itemize}
        \item<2-> Trouver son prédecesseur dans $\trm{renIds}_{A2}$ : $\coloridone\id{i}{B1}{0}\id{f}{A1}{0}$
        \item<3-> Trouver son équivalent à l'époque cible $\epoch{A2}$ : $\coloridtwo\id{i}{A2}{1}$
        \item<4-> Préfixer {$\coloridthree\id{i}{B1}{0}\id{m}{B2}{0}$} par ce dernier pour obtenir son équivalent à $\epoch{A2}$ : $\coloridfour\id{i}{A2}{1}\id{i}{B1}{0}\id{m}{B2}{0}$
      \end{itemize}
    \end{block}
  \end{frame}
