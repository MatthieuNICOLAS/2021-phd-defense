\subsection{Opération \emph{rename}}

\begin{frame}[fragile]{Opération \ren}
  \begin{figure}
    \resizebox{\textwidth}{!}{
      \begin{tikzpicture}
        \newcommand\nodeh[1]{
            node[letter, label=#1:{$\id{i}{B1}{0}$}] {H}
        }
        \newcommand\nodee[1]{
            node[letter, fill=\colorblockone, label=#1:{$\coloridone\id{i}{B1}{0}\id{f}{A1}{0}$}] {E}
        }
        \newcommand\nodelo[1]{
            node[block, label=#1:{$\id{i}{B1}{1..2}$}] {LO}
        }
        \newcommand\renh[1]{
            node[letter, fill=\colorblocktwo, label=#1:{$\coloridtwo\id{i}{A2}{0}$}] {H}
        }
        \newcommand\rene[1]{
            node[letter, fill=\colorblocktwo, label=#1:{$\coloridtwo\id{i}{A2}{1}$}] {E}
        }
        \newcommand\renl[1]{
            node[letter, fill=\colorblocktwo, label=#1:{$\coloridtwo\id{i}{A2}{2}$}] {L}
        }
        \newcommand\reno[1]{
            node[letter, fill=\colorblocktwo, label=#1:{$\coloridtwo\id{i}{A2}{3}$}] {O}
        }
        \newcommand\renhelo[1]{
            node[block, fill=\colorblocktwo, label=#1:{$\coloridtwo\id{i}{A2}{0..3}$}] {HELO}
        }

        \newcommand\initialstate[3]{
            \path
            #1
            ++#2
            ++(0:0.5)
            ++(#3:0.5) \nodeh{-#3}
            ++(0:\widthletter) \nodee{#3}
            ++(0:\widthletter) \nodelo{-#3};
        }

        \newcommand\stateh[3]{
            \path
            #1
            ++#2
            ++(0:0.5)
            ++(#3:0.5) \renh{-#3};
        }

        \newcommand\statehe[3]{
          \path
          #1
          ++#2
          ++(0:0.5)
          ++(#3:0.5) \renh{-#3}
          ++(0:\widthletter) \rene{-#3};
        }

        \newcommand\statehel[3]{
          \path
          #1
          ++#2
          ++(0:0.5)
          ++(#3:0.5) \renh{-#3}
          ++(0:\widthletter) \rene{-#3}
          ++(0:\widthletter) \renl{-#3};
        }

        \newcommand\statehelo[3]{
          \path
          #1
          ++#2
          ++(0:0.5)
          ++(#3:0.5) \renh{-#3}
          ++(0:\widthletter) \rene{-#3}
          ++(0:\widthletter) \renl{-#3}
          ++(0:\widthletter) \reno{-#3};
        }

        \newcommand\finalstate[3]{
          \path
          #1
          ++#2
          ++(0:0.5)
          ++(#3:0.5) \renhelo{-#3};
        }

        \newcommand\offseta{ (90:0.7) }

        \path
            node {\textbf{A}}
            ++(0:0.5) node (a) {}
            +(0:11) node (a-end) {}
            +(0:1) node[point] (a-initial) {}
            +(0:6) node[point, label=-170:{$\trm{ren}()$}, label={[xshift=0pt]-10:{$\trm{ren}(A,2,\trm{renIds})$}}] (a-ren-a1) {}
            +(0:10) node (a-final) {};

        \initialstate{(a-initial)}{\offseta}{90};
        \only<4>{\stateh{(a-ren-a1)}{\offseta}{90}};
        \only<5>{\statehe{(a-ren-a1)}{\offseta}{90};}
        \only<6>{\statehel{(a-ren-a1)}{\offseta}{90};}
        \only<7>{\statehelo{(a-ren-a1)}{\offseta}{90};}
        \onslide<8->{\finalstate{(a-ren-a1)}{\offseta}{90};}

        \draw[dotted] (a) -- (a-initial) (a-final) -- (a-end);
        \draw[->, thick] (a-initial) --  (a-ren-a1) -- (a-final);
      \end{tikzpicture}
    }
  \end{figure}
  \begin{itemize}
    \item<2-> Génère nouvel identifiant pour le 1er élément : \onslide<3->{$\id{i}{B1}{0} \to \id{i}{A2}{0}$}
    \item<4-> Puis génère identifiants contigus pour éléments suivants : \onslide<5->{$\id{i}{A2}{1}$}\onslide<6->{, $\id{i}{A2}{2}$}\onslide<7->{, \dots}
  \end{itemize}

  \onslide<8->{
    \begin{center}
      \alert{Regroupe tous les éléments en 1 unique bloc}
    \end{center}
  }

  \onslide<9->{
    \begin{itemize}
      \item Stocke identifiants ($[\id{i}{B1}{0},\id{i}{B1}{0}\id{f}{A1}{0},\dots]$) de l'état d'origine : $\trm{renIds}$
    \end{itemize}
  }
\end{frame}
