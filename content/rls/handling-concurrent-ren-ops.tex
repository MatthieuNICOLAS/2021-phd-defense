\subsection{Gestion des opérations \ren concurrentes}

\begin{frame}[standout]
  Et en cas d'opérations \ren concurrentes ?
\end{frame}

\begin{frame}[fragile]{Opérations \ren concurrentes}
  \begin{figure}
    \resizebox{\textwidth}{!}{
      \begin{tikzpicture}
        \newcommand\nodeh[1]{
            node[letter, label=#1:{$\id{i}{B1}{0}$}] {H}
        }
        \newcommand\nodee[1]{
            node[letter, fill=\colorblockone, label=#1:{\coloridone$\id{i}{B1}{0}\id{f}{A1}{0}$}] {E}
        }
        \newcommand\nodelo[1]{
            node[block, label=#1:{$\id{i}{B1}{1..2}$}] {LO}
        }
        \newcommand\renhelo[1]{
            node[block, fill=\colorblocktwo, label=#1:{\coloridtwo$\id{i}{A2}{0..3}$}] {HELO}
        }
        \newcommand\nodel[1]{
            node[letter, fill=\colorblockthree,label=#1:{\coloridthree$\id{i}{B1}{0}\id{m}{B2}{0}$}] {L}
        }
        \newcommand\renhe[1]{
            node[block, fill=\colorblocktwo, label=#1:{\coloridtwo$\id{i}{A2}{0..1}$}] {HE}
        }
        \newcommand\renl[1]{
            node[letter, fill=\colorblockfour, label=#1:{\coloridfour$\id{i}{A2}{1}\id{i}{B1}{0}\id{m}{B2}{0}$}] {L}
        }
        \newcommand\renlo[1]{
            node[block, fill=\colorblocktwo, label=#1:{\coloridtwo$\id{i}{A2}{2..3}$}] {LO}
        }
        \newcommand\renhello[1]{
            node[block, fill=\colorblockfive, label=#1:{\coloridfive$\id{i}{B3}{0..4}$}] {HELLO}
        }

        \newcommand\initialstate[3]{
            \path
                #1
                ++#2
                ++(0:0.5)
                ++(#3:0.5) node[epoch] {$\epoch{0}$}
                ++(0:1.25 * \widthoriginepoch) \nodeh{-#3}
                ++(0:\widthletter) \nodee{#3}
                ++(0:\widthletter) \nodelo{-#3};
        }

        \newcommand\statehelo[3]{
            \path
                #1
                ++#2
                ++(0:0.5)
                ++(#3:0.5) node[epoch] {$\epoch{A2}$}
                ++(0:1.18 * \widthepoch) \renhelo{-#3};
        }

        \newcommand\insl[3]{
            \path
            #1
            ++#2
            ++(0:0.5)
            ++(#3:0.5) node[epoch] {$\epoch{0}$}
            ++(0:1.25 * \widthoriginepoch) \nodeh{-#3}
            ++(0:\widthletter) \nodee{#3}
            ++(0:\widthletter) \nodel{-#3}
            ++(0:\widthletter) \nodelo{#3};
        }

        \newcommand\finalstatea[3]{
            \path
                #1
                ++#2
                ++(0:0.5)
                ++(#3:0.5) node[epoch] {$\epoch{A2}$}
                ++(0:1.18 * \widthepoch) \renhe{-#3}
                ++(0:\widthblock) \renl{#3}
                ++(0:\widthletter) \renlo{-#3};
        }
        \newcommand\finalstateb[3]{
            \path
                #1
                ++#2
                ++(0:0.5)
                ++(#3:0.5) node[epoch] {$\epoch{B3}$}
                ++(0:1.18 * \widthepoch) \renhello{#3};
        }

        \newcommand\offseta{ (90:0.7) }
        \newcommand\offsetb{ (-90:0.7) }

        \path
            node {\textbf{A}}
            ++(0:0.5) node (a) {}
            +(0:20) node (a-end) {}
            +(0:1) node[point] (a-initial) {}
            +(0:7) node[point, label=-170:{$\trm{ren}()$}, label={[xshift=0pt]-10:{$\trm{ren}(\epoch{0}, A,2)$}}] (a-ren-A2) {}
            +(0:15) node[point] (a-recv-ins-l) {}
            +(0:19) node (a-final) {};

        \initialstate{(a-initial)}{\offseta}{90};
        \statehelo{(a-ren-A2)}{\offseta}{90};
        \finalstatea{(a-recv-ins-l)}{\offseta}{90};

        \draw[dotted] (a) -- (a-initial) (a-final) -- (a-end);
        \draw[->, thick] (a-initial) --  (a-ren-A2) -- (a-recv-ins-l) -- (a-final);

        \path
            ++(270:3) node {\textbf{B}}
            ++(0:0.5) node (b) {}
            +(0:20) node (b-end) {}
            +(0:1) node[point] (b-initial) {}
            +(0:7) node[point, label=170:{$\trm{ins}(E \prec L \prec O)$}, label={[xshift=45pt]10:{$\trm{ins}(\epoch{0}, {\coloridthree\id{i}{B1}{0}\id{m}{B2}{0}},L)$}}] (b-ins-l) {}
            +(0:15) node[point, label=170:{$\trm{ren}()$}, label={[xshift=0pt]10:{$\trm{ren}(\epoch{0}, B,3)$}}] (b-ren-B3) {}
            +(0:19) node (b-final) {};


        \initialstate{(b-initial)}{\offsetb}{-90};
        \insl{(b-ins-l)}{\offsetb}{-90};
        \finalstateb{(b-ren-B3)}{\offsetb}{-90};

        \draw[dotted] (b) -- (b-initial) (b-final) -- (b-end);
        \draw[->, thick] (b-initial) --  (b-ins-l) -- (b-ren-B3) -- (b-final);

        \draw[->, dashed, shorten >= 1] (b-ins-l) -- (a-recv-ins-l);
      \end{tikzpicture}
    }
  \end{figure}
  \begin{itemize}
    \item Apparition d'une divergence
  \end{itemize}
  \alert{Besoin d'un mécanisme de résolution de conflits pour faire converger les noeuds}
\end{frame}

\begin{frame}{Résolution de conflits entre opérations \ren concurrentes}
  \begin{itemize}
    \item Opérations \ren sont des opérations systèmes, \ie pas d'intention utilisateur
    \item Peut ne pas appliquer les effets de certaines d'entre elles
  \end{itemize}
  \begin{block}{Intuition}
    \begin{itemize}
      \item Choisir une époque comme époque cible
      \item Calculer chemin entre époque courante et époque cible, et notamment leur Plus Proche Ancêtre Commun (PPAC)
      \item Annuler l'effet des opérations \ren de l'époque courante au PPAC
      \item Appliquer l'effet des opérations \ren du PPAC à l'époque cible
    \end{itemize}
  \end{block}
\end{frame}

\begin{frame}{Choisir une époque comme époque cible}
  \begin{figure}
    \resizebox{\textwidth}{!}{
      \begin{tikzpicture}
        \path
            node {\textbf{A}}
            ++(0:0.5) node (a) {}
            +(0:14) node (a-end) {}
            +(0:1) node (a-initial) {}
            +(0:3) node[point, label=above:{$\trm{ren}(\epoch{0}, A,2)$}] (a-ren-A2) {}
            +(0:13) node (a-final) {};

        \draw[dotted] (a) -- (a-initial) (a-final) -- (a-end);
        \draw[->, thick] (a-initial) --  (a-ren-A2) -- (a-final);

        \path
            ++(270:2) node {\textbf{B}}
            ++(0:0.5) node (b) {}
            +(0:14) node (b-end) {}
            +(0:1) node (b-initial) {}
            +(0:3) node[point, label=below:{$\trm{ren}(\epoch{0},B,3)$}] (b-ren-B3) {}
            +(0:11) node[point, label=below:{$\trm{ren}(\epoch{B3},B,7)$}] (b-ren-b7) {}
            +(0:13) node (b-final) {};

        \draw[dotted] (b) -- (b-initial) (b-final) -- (b-end);
        \draw[->, thick] (b-initial) -- (b-ren-B3) -- (b-ren-b7) -- (b-final);

        \path
            ++(270:4) node {\textbf{C}}
            ++(0:0.5) node (c) {}
            +(0:14) node (c-end) {}
            +(0:1) node (c-initial) {}
            +(0:7) node[point] (c-recv-ren-A2) {}
            +(0:9) node[point, label=below:{$\trm{ren}(\epoch{A2},C,6)$}] (c-ren-c6) {}
            +(0:13) node (c-final) {};

        \draw[dotted] (c) -- (c-initial) (c-final) -- (c-end);
        \draw[->, thick] (c-initial) -- (c-recv-ren-A2) -- (c-ren-c6) -- (c-final);

        \draw[->, dashed, shorten >= 1] (a-ren-A2) -- (c-recv-ren-A2);
      \end{tikzpicture}
    }
  \end{figure}
  \begin{columns}
    \begin{column}{0.3 \textwidth}
      \begin{figure}
        \resizebox{\columnwidth}{!}{
          \begin{tikzpicture}[scale=0.8,every node/.style={scale=0.3}]
            \path
              node[op] (e0) {$\epoch{0}$}
              ++(225:0.7) node[op] (eA2) {$\epoch{A2}$}
              ++(270:0.7) node[op] (eC6) {$\epoch{C6}$};
            \path
              ++(315:0.7) node[op] (eB3) {$\epoch{B3}$}
              ++(270:0.7) node[op, red] (eB7) {$\epoch{B7}$};

            \draw[->] (e0) edge (eA2) (eA2) edge (eC6) (e0) edge (eB3) (eB3) edge (eB7);
            \draw[->, dashed, red] (eB7.135) -- (eB3.225) (eB3.180) -- (eC6.45) -- (eA2.315) (eA2.0) -- (e0.270);
          \end{tikzpicture}
        }
      \end{figure}
    \end{column}
    \begin{column}{0.7 \textwidth}
      \begin{itemize}
        \item Doit définir relation \emph{priority}, notée $\lepoch$, ordre strict total sur les époques
        \item Utilise ordre lexicographique sur chemins des époques dans l'arbre
      \end{itemize}
    \end{column}
  \end{columns}
\end{frame}

\begin{frame}{Annuler l'effet d'une opération \ren}
  \begin{itemize}
    \item Propose un nouvel algorithme, \emph{revertRenameId}
    \item Permet d'obtenir l'identifiant correspondant à l'époque parente
  \end{itemize}
  \begin{block}{Intuition}
    \begin{enumerate}
      \item $\trm{id}$ fait partie des identifiants renommés : doit retourner son ancienne valeur
      \item $\trm{id}$ a potentiellement été inséré en concurrence : doit restaurer son ancienne valeur
      \item $\trm{id}$ a été inséré après le renommage : doit retourner une valeur qui préserve l'ordre
    \end{enumerate}
  \end{block}
\end{frame}

\begin{frame}[fragile]{Opérations \ren concurrentes}
  \begin{figure}
    \resizebox{\textwidth}{!}{
      \begin{tikzpicture}
        \newcommand\nodeh[1]{
          node[letter, label=#1:{$\id{i}{B1}{0}$}] {H}
        }
        \newcommand\nodee[1]{
          node[letter, fill=\colorblockone, label=#1:{\coloridone$\id{i}{B1}{0}\id{f}{A1}{0}$}] {E}
        }
        \newcommand\nodelo[1]{
          node[block, label=#1:{$\id{i}{B1}{1..2}$}] {LO}
        }
        \newcommand\renhelo[1]{
          node[block, fill=\colorblocktwo, label=#1:{\coloridtwo$\id{i}{A2}{0..3}$}] {HELO}
        }
        \newcommand\nodel[1]{
          node[letter, fill=\colorblockthree,label=#1:{\coloridthree$\id{i}{B1}{0}\id{m}{B2}{0}$}] {L}
        }
        \newcommand\renhe[1]{
          node[block, fill=\colorblocktwo, label=#1:{\coloridtwo$\id{i}{A2}{0..1}$}] {HE}
        }
        \newcommand\renl[1]{
          node[letter, fill=\colorblockfour, label=#1:{\coloridfour$\id{i}{A2}{1}\id{i}{B1}{0}\id{m}{B2}{0}$}] {L}
        }
        \newcommand\renlo[1]{
          node[block, fill=\colorblocktwo, label=#1:{\coloridtwo$\id{i}{A2}{2..3}$}] {LO}
        }
        \newcommand\renhello[1]{
          node[block, fill=\colorblockfive, label=#1:{\coloridfive$\id{i}{B3}{0..4}$}] {HELLO}
        }

        \newcommand\insl[3]{
          \path
            #1
            ++#2
            ++(0:0.5)
            ++(#3:0.5) node[epoch] {$\epoch{0}$}
            ++(0:1.25 * \widthoriginepoch) \nodeh{-#3}
            ++(0:\widthletter) \nodee{#3}
            ++(0:\widthletter) \nodel{-#3}
            ++(0:\widthletter) \nodelo{#3};
        }

        \newcommand\recvrenl[3]{
          \path
            #1
            ++#2
            ++(0:0.5)
            ++(#3:0.5) node[epoch] {$\epoch{A2}$}
            ++(0:1.18 * \widthepoch) \renhe{-#3}
            ++(0:\widthblock) \renl{#3}
            ++(0:\widthletter) \renlo{-#3};
        }
        \newcommand\finalstatea[3]{
          \path
            #1
            ++#2
            ++(0:0.5)
            ++(#3:0.5) node[epoch] {$\epoch{A2}$}
            ++(0:1.18 * \widthepoch) \renhe{-#3}
            ++(0:\widthblock) \renl{#3}
            ++(0:\widthletter) \renlo{-#3}
            ++(0:\widthblock) node (eA2-right) {}
            ++(0:1.5) node[epoch] (e0-left) {$\epoch{0}$}
            ++(0:1.25 * \widthoriginepoch) \nodeh{-#3}
            ++(0:\widthletter) \nodee{#3}
            ++(0:\widthletter) \nodel{-#3}
            ++(0:\widthletter) \nodelo{#3}
            ++(0:\widthblock) node (e0-right) {}
            ++(0:1.5) node[epoch] (eB3-left) {$\epoch{B3}$}
            ++(0:1.18 * \widthepoch) \renhello{#3};

            \draw[->, loosely dash dot, shorten >= 1] (eA2-right) --  node[above, align=center]{\emph{revert}\\ \emph{to} $\epoch{0}$} (e0-left);
            \draw[->, loosely dash dot, shorten >= 1] (e0-right) --  node[above, align=center]{$\trm{rename}$\\ \emph{to} $\epoch{B3}$} (eB3-left);
        }
        \newcommand\finalstateb[3]{
          \path
            #1
            ++#2
            ++(0:0.5)
            ++(#3:0.5) node[epoch] {$\epoch{B3}$}
            ++(0:1.18 * \widthepoch) \renhello{#3};
        }

        \newcommand\offseta{ (90:0.7) }
        \newcommand\offsetb{ (-90:0.7) }

        \path
          node {\textbf{A}}
          ++(0:0.5) node (a) {}
          +(0:20) node (a-end) {}
          +(0:1) node[point] (a-recv-ins-l) {}
          +(0:6) node[point] (a-recv-ren-B3) {}
          +(0:19) node (a-final) {};

        \recvrenl{(a-recv-ins-l)}{\offseta}{90};
        \finalstatea{(a-recv-ren-B3)}{\offseta}{90};

        \draw[dotted] (a) -- (a-recv-ins-l) (a-final) -- (a-end);
        \draw[->, thick] (a-recv-ins-l) -- (a-recv-ren-B3) -- (a-final);

        \path
          ++(270:3) node {\textbf{B}}
          ++(0:0.5) node (b) {}
          +(0:20) node (b-end) {}
          +(0:1) node[point, label=170:{$\trm{ren}()$}, label={[xshift=30pt]10:{$\trm{ren}(\epoch{0}, B,3)$}}] (b-ren-B3) {}
          +(0:19) node (b-final) {};

        \finalstateb{(b-ren-B3)}{\offsetb}{-90};

        \draw[dotted] (b) -- (b-ren-B3) (b-final) -- (b-end);
        \draw[->, thick] (b-ren-B3) -- (b-final);

        \draw[->, dashed, shorten >= 1] (b-ren-B3) -- (a-recv-ren-B3);
    \end{tikzpicture}
    }
  \end{figure}
  \begin{columns}
    \begin{column}{0.3 \textwidth}
      \begin{figure}
        \resizebox{\columnwidth}{!}{
          \begin{tikzpicture}[scale=0.8,every node/.style={scale=0.3}]
            \path
              node[op] (e0) {$\epoch{0}$}
              +(225:0.7) node[op] (eA2) {$\epoch{A2}$}
              +(315:0.7) node[op, red] (eB3) {$\epoch{B3}$};

            \draw[->] (e0) edge (eA2) (eA2) (e0) edge (eB3);
            \draw[->, dashed, red] (eB3.180) -- (eA2.0) (eA2.0) -- (e0.270);
          \end{tikzpicture}
        }
      \end{figure}
    \end{column}
    \begin{column}{0.7 \textwidth}
      \begin{itemize}
        \item TODO: Illustrer choix de l'époque cible, cas 1 et cas 2
      \end{itemize}
    \end{column}
  \end{columns}
\end{frame}

% \begin{frame}{Applying concurrent \emph{rename} operations}
%   \begin{figure}
%     \centering
%     \includegraphics<1>[width=\columnwidth]{../2021-phd-day-figures/resolving-concurrent-rename/1/figure.pdf}
%     \includegraphics<2>[width=\columnwidth]{../2021-phd-day-figures/resolving-concurrent-rename/2/figure.pdf}
%     \includegraphics<3>[width=\columnwidth]{../2021-phd-day-figures/resolving-concurrent-rename/3/figure.pdf}
%     \caption{Applying a concurrent \emph{rename} operation}
%   \end{figure}
%   \vspace{-3mm}
%   \begin{itemize}
%     \item<2-> Revert state to equivalent one at LCA epoch
%     \item<3> Apply then \emph{rename} operations leading to target epoch
%   \end{itemize}
% \end{frame}

% \begin{frame}{Downsides}
%   \begin{block}{Need to store former state until no more concurrent operations}
%     \begin{itemize}
%       \item Can garbage collect it once the \emph{rename} operation is causally stable \footcite{10.1007/978-3-662-43352-2_11}
%       \item Can offload it to the disk meanwhile
%     \end{itemize}
%   \end{block}

%   \begin{block}{Need to propagate former state to other nodes}
%     \begin{itemize}
%       \item Can compress the operation to minimise bandwidth consumption
%       \item Can trigger \emph{rename} operations at a given number of blocks
%     \end{itemize}
%   \end{block}
% \end{frame}
