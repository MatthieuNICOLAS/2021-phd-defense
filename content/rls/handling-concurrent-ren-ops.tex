\subsection{Gestion des opérations \ren concurrentes}

\begin{frame}[standout]
  Et en cas d'opérations \ren concurrentes ?
\end{frame}

\begin{frame}[fragile]{Opérations \ren concurrentes}
  \begin{figure}
    \resizebox{\textwidth}{!}{
      \begin{tikzpicture}
        \newcommand\nodeh[1]{
            node[letter, label=#1:{$\id{i}{B1}{0}$}] {H}
        }
        \newcommand\nodee[1]{
            node[letter, fill=\colorblockone, label=#1:{\coloridone$\id{i}{B1}{0}\id{f}{A1}{0}$}] {E}
        }
        \newcommand\nodelo[1]{
            node[block, label=#1:{$\id{i}{B1}{1..2}$}] {LO}
        }
        \newcommand\renhelo[1]{
            node[block, fill=\colorblocktwo, label=#1:{\coloridtwo$\id{i}{A2}{0..3}$}] {HELO}
        }
        \newcommand\nodel[1]{
            node[letter, fill=\colorblockthree,label=#1:{\coloridthree$\id{i}{B1}{0}\id{m}{B2}{0}$}] {L}
        }
        \newcommand\renhe[1]{
            node[block, fill=\colorblocktwo, label=#1:{\coloridtwo$\id{i}{A2}{0..1}$}] {HE}
        }
        \newcommand\renl[1]{
            node[letter, fill=\colorblockfour, label=#1:{\coloridfour$\id{i}{A2}{1}\id{i}{B1}{0}\id{m}{B2}{0}$}] {L}
        }
        \newcommand\renlo[1]{
            node[block, fill=\colorblocktwo, label=#1:{\coloridtwo$\id{i}{A2}{2..3}$}] {LO}
        }
        \newcommand\renhello[1]{
            node[block, fill=\colorblockfive, label=#1:{\coloridfive$\id{i}{B3}{0..4}$}] {HELLO}
        }

        \newcommand\statehelo[3]{
            \path
                #1
                ++#2
                ++(0:0.5)
                ++(#3:0.5) node[epoch] {$\epoch{A2}$}
                ++(0:1.18 * \widthepoch) \renhelo{-#3};
        }

        \newcommand\insl[3]{
            \path
            #1
            ++#2
            ++(0:0.5)
            ++(#3:0.5) node[epoch] {$\epoch{0}$}
            ++(0:1.25 * \widthoriginepoch) \nodeh{-#3}
            ++(0:\widthletter) \nodee{#3}
            ++(0:\widthletter) \nodel{-#3}
            ++(0:\widthletter) \nodelo{#3};
        }

        \newcommand\finalstatea[3]{
            \path
                #1
                ++#2
                ++(0:0.5)
                ++(#3:0.5) node[epoch] {$\epoch{A2}$}
                ++(0:1.18 * \widthepoch) \renhe{-#3}
                ++(0:\widthblock) \renl{#3}
                ++(0:\widthletter) \renlo{-#3};
        }
        \newcommand\finalstateb[3]{
            \path
                #1
                ++#2
                ++(0:0.5)
                ++(#3:0.5) node[epoch] {$\epoch{B3}$}
                ++(0:1.18 * \widthepoch) \renhello{#3};
        }

        \newcommand\offseta{ (90:0.7) }
        \newcommand\offsetb{ (-90:0.7) }

        \path
          node {\textbf{A}}
          ++(0:0.5) node (a) {}
          +(0:13) node (a-end) {}
          +(0:1) node[point] (a-initial) {}
          +(0:7) node[point] (a-recv-ins-l) {}
          +(0:12) node (a-final) {};

        \statehelo{(a-initial)}{\offseta}{90};
        \finalstatea{(a-recv-ins-l)}{\offseta}{90};

        \draw[dotted] (a) -- (a-initial) (a-final) -- (a-end);
        \draw[->, thick] (a-initial) -- (a-recv-ins-l) -- (a-final);

        \path
          ++(270:2) node {\textbf{B}}
          ++(0:0.5) node (b) {}
          +(0:13) node (b-end) {}
          +(0:1) node[point, label={[xshift=45pt]10:{$\trm{ins}(\epoch{0}, {\coloridthree\id{i}{B1}{0}\id{m}{B2}{0}},L)$}}] (b-initial) {}
          +(0:7) node (b-ren-b3) {}
          +(0:12) node (b-final) {};

        \insl{(b-initial)}{\offsetb}{-90};

        \draw[dotted] (b) -- (b-initial) (b-final) -- (b-end);
        \draw[->, dashed, shorten >= 1] (b-initial) -- (a-recv-ins-l);

        \draw<1>[->, thick] (b-initial) -- (b-final);
        \onslide<2->{
          \path
            (b-ren-b3) node[point, label=170:{$\trm{ren}()$}] {};
          \draw[->, thick] (b-initial) -- (b-ren-b3) -- (b-final);
        }
        \onslide<3->{
          \finalstateb{(b-ren-b3)}{\offsetb}{-90};
        }
      \end{tikzpicture}
    }
  \end{figure}
  \begin{center}
    \onslide<4->{Comment faire converger les noeuds ?}

    \onslide<5>{\alert{Besoin d'un mécanisme additionnel de résolution de conflits}}
  \end{center}
\end{frame}

\begin{frame}{Résolution de conflits entre opérations \ren concurrentes}
  \metroset{block=transparent}
  \begin{block}{Observation}
    \begin{itemize}
      \item Opérations \ren sont des opérations systèmes\dots
      \item \dots pas des opérations utilisateur-rices
    \end{itemize}
  \end{block}
  \pause
  \begin{block}{Proposition}
    \begin{itemize}
      \item Considérer une opération \ren comme prioritaire\dots
      \item \dots et ignorer les opérations \ren en conflit avec elle
    \end{itemize}
  \end{block}
\end{frame}

\begin{frame}{Algorithme d'intégration d'une opération \ren}
  \begin{block}{Intuition}
    \begin{enumerate}
      \item<2-> \only<8,9>{\color{gray}}Ajouter l'époque créée par l'opération \ren à l'ensemble des époques connues
      \item<3->
        \only<3,4,5,6,7,8>{Choisir entre époque courante et nouvelle époque \alert{l'époque cible}}
        \only<9>{\alert{Choisir entre époque courante et nouvelle époque l'époque cible}}
      \item<4-> Si changement d'époque cible
      \begin{enumerate}
        \item<5-> Calculer chemin entre époque courante et époque cible, et notamment leur Plus Proche Ancêtre Commun (PPAC)
        \item<6-> Annuler l'effet des opérations \ren de l'époque courante au PPAC
        \item<7-> Appliquer l'effet des opérations \ren du PPAC à l'époque cible
      \end{enumerate}
    \end{enumerate}
  \end{block}
\end{frame}

\begin{frame}{Choisir une époque comme époque cible}
  \metroset{block=transparent}
  \begin{figure}
    \resizebox{\textwidth}{!}{
      \begin{tikzpicture}
        \path
            node {\textbf{A}}
            ++(0:0.5) node (a) {}
            +(0:14) node (a-end) {}
            +(0:1) node (a-initial) {}
            +(0:3) node (a-ren-A2) {}
            +(0:13) node (a-final) {};

        \draw[dotted] (a) -- (a-initial) (a-final) -- (a-end);

        \path
            ++(270:2) node {\textbf{B}}
            ++(0:0.5) node (b) {}
            +(0:14) node (b-end) {}
            +(0:1) node (b-initial) {}
            +(0:3) node (b-ren-B3) {}
            +(0:11) node (b-ren-b7) {}
            +(0:13) node (b-final) {};

        \draw[dotted] (b) -- (b-initial) (b-final) -- (b-end);

        \path
            ++(270:4) node {\textbf{C}}
            ++(0:0.5) node (c) {}
            +(0:14) node (c-end) {}
            +(0:1) node (c-initial) {}
            +(0:7) node (c-recv-ren-A2) {}
            +(0:9) node (c-ren-c6) {}
            +(0:13) node (c-final) {};

        \draw[dotted] (c) -- (c-initial) (c-final) -- (c-end);

        \draw<1>[->, thick] (a-initial) -- (a-final);
        \draw<1,2>[->, thick] (b-initial) -- (b-final);
        \draw<1,2,3>[->, thick] (c-initial) -- (c-final);
        \onslide<2->{
          \path
            (a-ren-A2) node[point, label=above:{$\trm{ren}(\epoch{0}, A,2,\renids{A2})$}] {};
          \draw<2->[->, thick] (a-initial) --  (a-ren-A2) -- (a-final);
        }
        \draw<3-5>[->, thick] (b-initial) -- (b-ren-B3) -- (b-final);
        \onslide<3-> {
          \path
            (b-ren-B3) node[point, label=below:{$\trm{ren}(\epoch{0},B,3,\renids{B3})$}] {};
        }
        \draw<4>[->, thick] (c-initial) -- (c-recv-ren-A2) -- (c-final);
        \onslide<4-> {
          \path
            (c-recv-ren-A2) node[point] {};
          \draw[->, dashed, shorten >= 1] (a-ren-A2) -- (c-recv-ren-A2);
        }
        \onslide<5-> {
          \path
            (c-ren-c6) node[point, label=below:{$\trm{ren}(\epoch{A2},C,6,\renids{C6})$}] {};
          \draw[->, thick] (c-initial) -- (c-recv-ren-A2) -- (c-ren-c6) -- (c-final);
        }
        \onslide<6-> {
          \path
            (b-ren-b7) node[point, label=below:{$\trm{ren}(\epoch{B3},B,7,\renids{B7})$}] {};
          \draw[->, thick] (b-initial) -- (b-ren-B3) -- (b-ren-b7) -- (b-final);
        }

      \end{tikzpicture}
    }
  \end{figure}
  \begin{columns}
    \begin{column}{0.3 \textwidth}
      \vspace{-1.3em}
      \begin{block}{Arbre des époques}
        \begin{figure}
          \resizebox{\columnwidth}{!}{
            \begin{tikzpicture}[scale=0.8,every node/.style={scale=0.3}]
              \path
                node[op] (e0) {$\epoch{0}$}
                ++(225:0.7) node (eA2) {}
                ++(270:0.7) node (eC6) {};
              \path
                ++(315:0.7) node (eB3) {}
                ++(270:0.7) node (eB7) {};

              \onslide<2->{
                \path
                  (eA2) node[op] (eA2-node) {$\epoch{A2}$};
                \draw[->] (e0) edge (eA2-node);
              }

              \onslide<3->{
                \path
                  (eB3) node[op] (eB3-node) {$\epoch{B3}$};
                \draw[->] (e0) edge (eB3-node);
              }

              \onslide<5->{
                \path
                  (eC6) node[op] (eC6-node) {$\epoch{C6}$};
                \draw[->] (eA2-node) edge (eC6-node);
              }

              \onslide<6-11>{
                \path
                  (eB7) node[op] (eB7-node-1) {$\epoch{B7}$};
                \draw[->] (eB3-node) edge (eB7-node-1);
              }

              \onslide<12>{
                \path
                  (eB7) node[op, targetepoch] (eB7-node-2) {$\epoch{B7}$};
                \draw[->] (eB3-node) edge (eB7-node-2);
                \draw[->, dashed, targetepoch] (eB7-node-2.135) -- (eB3-node.225) (eB3-node.180) -- (eC6-node.45) -- (eA2-node.315) (eA2-node.0) -- (e0.270);
              }
            \end{tikzpicture}
          }
        \end{figure}
      \end{block}
    \end{column}
    \begin{column}{0.7 \textwidth}
      \vspace{-2.5em}
      \begin{block}<7->{Comment choisir ?}
        \begin{itemize}
          \item<8-> Définit relation \emph{priority}, notée $\lepoch$, ordre strict total sur les époques
          \item<9-> Utilise \alert{ordre lexicographique sur chemins} des époques dans l'arbre
        \end{itemize}
      \end{block}
      \vspace{-1em}
      \begin{block}<10->{Exemple}
        \begin{equation*}
          \only<10,11,12>{\epoch{0} < \epoch{0}\epoch{A2}}
          \only<11,12>{< \epoch{0}\epoch{A2}\epoch{C6}}
          \only<12>{< \epoch{0}\epoch{B3}\epoch{B7}}
        \end{equation*}
      \end{block}
    \end{column}
  \end{columns}
\end{frame}

\begin{frame}[fragile]{Exemple - Calculs des transformations à effectuer}
  \metroset{block=transparent}
  \begin{figure}
    \resizebox{\textwidth}{!}{
      \begin{tikzpicture}
        \newcommand\nodeh[1]{
          node[letter, label=#1:{$\id{i}{B1}{0}$}] {H}
        }
        \newcommand\nodee[1]{
          node[letter, fill=\colorblockone, label=#1:{\coloridone$\id{i}{B1}{0}\id{f}{A1}{0}$}] {E}
        }
        \newcommand\nodelo[1]{
          node[block, label=#1:{$\id{i}{B1}{1..2}$}] {LO}
        }
        \newcommand\renhelo[1]{
          node[block, fill=\colorblocktwo, label=#1:{\coloridtwo$\id{i}{A2}{0..3}$}] {HELO}
        }
        \newcommand\nodel[1]{
          node[letter, fill=\colorblockthree,label=#1:{\coloridthree$\id{i}{B1}{0}\id{m}{B2}{0}$}] {L}
        }
        \newcommand\renhe[1]{
          node[block, fill=\colorblocktwo, label=#1:{\coloridtwo$\id{i}{A2}{0..1}$}] {HE}
        }
        \newcommand\renl[1]{
          node[letter, fill=\colorblockfour, label=#1:{\coloridfour$\id{i}{A2}{1}\id{i}{B1}{0}\id{m}{B2}{0}$}] {L}
        }
        \newcommand\renlo[1]{
          node[block, fill=\colorblocktwo, label=#1:{\coloridtwo$\id{i}{A2}{2..3}$}] {LO}
        }
        \newcommand\renhello[1]{
          node[block, fill=\colorblockfive, label=#1:{\coloridfive$\id{i}{B3}{0..4}$}] {HELLO}
        }

        \newcommand\insl[3]{
          \path
            #1
            ++#2
            ++(0:0.5)
            ++(#3:0.5) node[epoch] {$\epoch{0}$}
            ++(0:1.25 * \widthoriginepoch) \nodeh{-#3}
            ++(0:\widthletter) \nodee{#3}
            ++(0:\widthletter) \nodel{-#3}
            ++(0:\widthletter) \nodelo{#3};
        }

        \newcommand\recvrenl[3]{
          \path
            #1
            ++#2
            ++(0:0.5)
            ++(#3:0.5) node[epoch] {$\epoch{A2}$}
            ++(0:1.18 * \widthepoch) \renhe{-#3}
            ++(0:\widthblock) \renl{#3}
            ++(0:\widthletter) \renlo{-#3};
        }
        \newcommand\finalstatea[3]{
          \path
            #1
            ++#2
            ++(0:0.5)
            ++(#3:0.5) node[epoch] {$\epoch{A2}$}
            ++(0:1.18 * \widthepoch) \renhe{-#3}
            ++(0:\widthblock) \renl{#3}
            ++(0:\widthletter) \renlo{-#3}
            ++(0:\widthblock) node (eA2-right) {};
        }
        \newcommand\totoa[3]{
          \path
            #1
            ++#2
            ++(0:0.5)
            ++(#3:0.5) node[epoch] {$\epoch{A2}$}
            ++(0:1.18 * \widthepoch) \renhe{-#3}
            ++(0:\widthblock) \renl{#3}
            ++(0:\widthletter) \renlo{-#3}
            ++(0:\widthblock) node (eA2-right) {}
            +(0:2) node (e0-left) {}
            +(0:3) node {?};

            \draw[->, loosely dash dot, shorten >= 1] (eA2-right) --  node[above, align=center]{\emph{revert}\\ \emph{to} $\epoch{0}$} (e0-left);
        }
        \newcommand\totob[3]{
          \path
            #1
            ++#2
            ++(0:0.5)
            ++(#3:0.5) node[epoch] {$\epoch{A2}$}
            ++(0:1.18 * \widthepoch) \renhe{-#3}
            ++(0:\widthblock) \renl{#3}
            ++(0:\widthletter) \renlo{-#3}
            ++(0:\widthblock) node (eA2-right) {}
            +(0:2) node (e0-left) {}
            +(0:3) node {?}
            +(0:4) node (e0-right) {}
            ++(0:6) node[epoch] (eB3-left) {$\epoch{B3}$}
            ++(0:1.18 * \widthepoch) \renhello{#3};

            \draw[->, loosely dash dot, shorten >= 1] (eA2-right) --  node[above, align=center]{\emph{revert}\\ \emph{to} $\epoch{0}$} (e0-left);
            \draw[->, loosely dash dot, shorten >= 1] (e0-right) --  node[above, align=center]{$\trm{rename}$\\ \emph{to} $\epoch{B3}$} (eB3-left);
        }
        \newcommand\finalstateb[3]{
          \path
            #1
            ++#2
            ++(0:0.5)
            ++(#3:0.5) node[epoch] {$\epoch{B3}$}
            ++(0:1.18 * \widthepoch) \renhello{#3};
        }

        \newcommand\offseta{ (90:0.7) }
        \newcommand\offsetb{ (-90:0.7) }

        \path
          node {\textbf{A}}
          ++(0:0.5) node (a) {}
          +(0:20) node (a-end) {}
          +(0:1) node[point] (a-recv-ins-l) {}
          +(0:6) node (a-recv-ren-B3) {}
          +(0:19) node (a-final) {};

        \recvrenl{(a-recv-ins-l)}{\offseta}{90};

        \draw[dotted] (a) -- (a-recv-ins-l) (a-final) -- (a-end);

        \path
          ++(270:3) node {\textbf{B}}
          ++(0:0.5) node (b) {}
          +(0:20) node (b-end) {}
          +(0:1) node[point, label=170:{$\trm{ren}()$}, label={[xshift=30pt]10:{$\trm{ren}(\epoch{0}, B,3)$}}] (b-ren-B3) {}
          +(0:19) node (b-final) {};

        \finalstateb{(b-ren-B3)}{\offsetb}{-90};

        \draw[dotted] (b) -- (b-ren-B3) (b-final) -- (b-end);
        \draw[->, thick] (b-ren-B3) -- (b-final);

        \only<1>{
          \draw[->, thick] (a-recv-ins-l) -- (a-final);
        }
        \onslide<2->{
          \path
            (a-recv-ren-B3) node[point] {};
          \draw[->, thick] (a-recv-ins-l) -- (a-recv-ren-B3) -- (a-final);
          \draw[->, dashed, shorten >= 1] (b-ren-B3) -- (a-recv-ren-B3);
        }
        \onslide<2-4>{
          \finalstatea{(a-recv-ren-B3)}{\offseta}{90};
        }
        \only<5>{
          \totoa{(a-recv-ren-B3)}{\offseta}{90};
        }
        \only<6>{
          \totob{(a-recv-ren-B3)}{\offseta}{90};
        }
    \end{tikzpicture}
    }
  \end{figure}
  \begin{columns}
    \begin{column}{0.4 \textwidth}
      \begin{block}{Arbre des époques de A}
        \begin{figure}
          \resizebox{0.8 \columnwidth}{!}{
            \begin{tikzpicture}[scale=0.8,every node/.style={scale=0.3}]
              % \path
              %   +(0:1) node {}
              %   +(180:-1) node {};
              \path
                node[op] (e0) {$\epoch{0}$}
                +(225:0.7) node[op, currentepoch] (eA2) {$\epoch{A2}$}
                +(315:0.7) node (eB3) {};

              \draw[->] (e0) edge (eA2);

              \only<2> {
                \path
                  (eB3) node[op] (eB3-node-1) {$\epoch{B3}$};
                \draw[->] (e0) edge (eB3-node-1);
              }
              \onslide<3-> {
                \path
                  (eB3) node[op,targetepoch] (eB3-node-2) {$\epoch{B3}$};
                \draw[->] (e0) edge (eB3-node-2);
              }
              \onslide<5-> {
                \draw[->, densely dash dot, thin, gray] (eA2) edge[bend left] node[xshift=-2em]{$\trm{revert}$} (e0.west);
              }
              \only<6> {
                \draw[->, densely dash dot, thin, gray] (e0) edge[bend left] node[xshift=2em]{$\trm{apply}$} (eB3-node-2.north);
              }
            \end{tikzpicture}
          }
        \end{figure}
      \end{block}
    \end{column}
    \begin{column}{0.6 \textwidth}
      \begin{block}{Étapes}
        \begin{itemize}
          \item Époque courante : $\epoch{A2}$
          \item<3-> Époque cible : $\epoch{B3}$
          \item<4-> Plus Proche Ancêtre Commun : $\epoch{0}$
        \end{itemize}
      \end{block}
      \begin{center}
        \alert{
          \onslide<5->{Doit annuler $\epoch{A2}$}
          \only<6>{puis appliquer $\epoch{B3}$}
        }
      \end{center}
    \end{column}
  \end{columns}
\end{frame}

\begin{frame}{Algorithme d'intégration d'une opération \ren}
  \begin{block}{Intuition}
    \begin{enumerate}
      \item {\color{gray}Ajouter l'époque créée par l'opération \ren à l'ensemble des époques connues}
      \item {\color{gray}Choisir entre époque courante et nouvelle époque l'époque cible}
      \item {\color{gray}Si changement d'époque cible}
      \begin{enumerate}
        \item {\color{gray}Calculer chemin entre époque courante et époque cible, et notamment leur Plus Proche Ancêtre Commun (PPAC)}
        \item
          \only<1>{Annuler l'effet des opérations \ren de l'époque courante au PPAC}
          \only<2>{\alert{Annuler l'effet des opérations \ren de l'époque courante au PPAC}}
        \item Appliquer l'effet des opérations \ren du PPAC à l'époque cible
      \end{enumerate}
    \end{enumerate}
  \end{block}
\end{frame}

\begin{frame}{Annuler l'effet d'une opération \ren}
  \metroset{block=transparent}
  \begin{block}{Ajout d'un nouveau mécanisme de transformation}
    \begin{itemize}
      \item<2-> Prend la forme de l'algorithme \emph{revertRenameId}
      \item<3-> \alert{Exclure l'effet de l'opération \ren}
    \end{itemize}
  \end{block}
  \metroset{block=fill}
  \onslide<4->{
    \begin{block}{Intuition}
      \begin{enumerate}
        \item<5-> $\trm{id}$ fait partie des identifiants renommés : doit retourner son ancienne valeur
        \item<6->
          \only<6,7,8>{$\trm{id}$ a (potentiellement) été inséré en concurrence : doit restaurer sa (potentielle) ancienne valeur}
          \only<9>{\alert{$\trm{id}$ a (potentiellement) été inséré en concurrence : doit restaurer sa (potentielle) ancienne valeur}}
        \item<7-> $\trm{id}$ a été inséré après le renommage : doit retourner une valeur qui préserve l'ordre
      \end{enumerate}
    \end{block}
  }
  \onslide<8->{
    \begin{center}
      Distingue cas par \alert{filtrage par motif}
    \end{center}
  }
\end{frame}

\begin{frame}[fragile]{Opérations \ren concurrentes - Choix de l'époque cible}
  \begin{figure}
    \resizebox{\textwidth}{!}{
      \begin{tikzpicture}
        \newcommand\nodeh[1]{
          node[letter, label=#1:{$\id{i}{B1}{0}$}] {H}
        }
        \newcommand\nodee[1]{
          node[letter, fill=\colorblockone, label=#1:{\coloridone$\id{i}{B1}{0}\id{f}{A1}{0}$}] {E}
        }
        \newcommand\nodelo[1]{
          node[block, label=#1:{$\id{i}{B1}{1..2}$}] {LO}
        }
        \newcommand\renhelo[1]{
          node[block, fill=\colorblocktwo, label=#1:{\coloridtwo$\id{i}{A2}{0..3}$}] {HELO}
        }
        \newcommand\nodel[1]{
          node[letter, fill=\colorblockthree,label=#1:{\coloridthree$\id{i}{B1}{0}\id{m}{B2}{0}$}] {L}
        }
        \newcommand\renhe[1]{
          node[block, fill=\colorblocktwo, label=#1:{\coloridtwo$\id{i}{A2}{0..1}$}] {HE}
        }
        \newcommand\renl[1]{
          node[letter, fill=\colorblockfour, label=#1:{\coloridfour$\id{i}{A2}{1}\id{i}{B1}{0}\id{m}{B2}{0}$}] {L}
        }
        \newcommand\renlo[1]{
          node[block, fill=\colorblocktwo, label=#1:{\coloridtwo$\id{i}{A2}{2..3}$}] {LO}
        }
        \newcommand\renhello[1]{
          node[block, fill=\colorblockfive, label=#1:{\coloridfive$\id{i}{B3}{0..4}$}] {HELLO}
        }

        \newcommand\insl[3]{
          \path
            #1
            ++#2
            ++(0:0.5)
            ++(#3:0.5) node[epoch] {$\epoch{0}$}
            ++(0:1.25 * \widthoriginepoch) \nodeh{-#3}
            ++(0:\widthletter) \nodee{#3}
            ++(0:\widthletter) \nodel{-#3}
            ++(0:\widthletter) \nodelo{#3};
        }

        \newcommand\recvrenl[3]{
          \path
            #1
            ++#2
            ++(0:0.5)
            ++(#3:0.5) node[epoch] {$\epoch{A2}$}
            ++(0:1.18 * \widthepoch) \renhe{-#3}
            ++(0:\widthblock) \renl{#3}
            ++(0:\widthletter) \renlo{-#3};
        }
        \newcommand\finalstatea[3]{
          \path
            #1
            ++#2
            ++(0:0.5)
            ++(#3:0.5) node[epoch] {$\epoch{A2}$}
            ++(0:1.18 * \widthepoch) \renhe{-#3}
            ++(0:\widthblock) \renl{#3}
            ++(0:\widthletter) \renlo{-#3}
            ++(0:\widthblock) node (eA2-right) {}
            ++(0:1.5) node[epoch] (e0-left) {$\epoch{0}$}
            ++(0:1.25 * \widthoriginepoch) \nodeh{-#3}
            ++(0:\widthletter) \nodee{#3}
            ++(0:\widthletter) \nodel{-#3}
            ++(0:\widthletter) \nodelo{#3}
            ++(0:\widthblock) node (e0-right) {}
            ++(0:1.5) node[epoch] (eB3-left) {$\epoch{B3}$}
            ++(0:1.18 * \widthepoch) \renhello{#3};

            \draw[->, loosely dash dot, shorten >= 1] (eA2-right) --  node[above, align=center]{\emph{revert}\\ \emph{to} $\epoch{0}$} (e0-left);
            \draw[->, loosely dash dot, shorten >= 1] (e0-right) --  node[above, align=center]{$\trm{rename}$\\ \emph{to} $\epoch{B3}$} (eB3-left);
        }
        \newcommand\finalstateb[3]{
          \path
            #1
            ++#2
            ++(0:0.5)
            ++(#3:0.5) node[epoch] {$\epoch{B3}$}
            ++(0:1.18 * \widthepoch) \renhello{#3};
        }

        \newcommand\offseta{ (90:0.7) }
        \newcommand\offsetb{ (-90:0.7) }

        \path
          node {\textbf{A}}
          ++(0:0.5) node (a) {}
          +(0:20) node (a-end) {}
          +(0:1) node[point] (a-recv-ins-l) {}
          +(0:6) node[point] (a-recv-ren-B3) {}
          +(0:19) node (a-final) {};

        \recvrenl{(a-recv-ins-l)}{\offseta}{90};
        \finalstatea{(a-recv-ren-B3)}{\offseta}{90};

        \draw[dotted] (a) -- (a-recv-ins-l) (a-final) -- (a-end);
        \draw[->, thick] (a-recv-ins-l) -- (a-recv-ren-B3) -- (a-final);

        \path
          ++(270:3) node {\textbf{B}}
          ++(0:0.5) node (b) {}
          +(0:20) node (b-end) {}
          +(0:1) node[point, label=170:{$\trm{ren}()$}, label={[xshift=30pt]10:{$\trm{ren}(\epoch{0}, B,3)$}}] (b-ren-B3) {}
          +(0:19) node (b-final) {};

        \finalstateb{(b-ren-B3)}{\offsetb}{-90};

        \draw[dotted] (b) -- (b-ren-B3) (b-final) -- (b-end);
        \draw[->, thick] (b-ren-B3) -- (b-final);

        \draw[->, dashed, shorten >= 1] (b-ren-B3) -- (a-recv-ren-B3);
    \end{tikzpicture}
    }
  \end{figure}
  \begin{columns}
    \begin{column}{0.3 \textwidth}
      \begin{figure}
        \resizebox{\columnwidth}{!}{
          \begin{tikzpicture}[scale=0.8,every node/.style={scale=0.3}]
            \path
              node[op] (e0) {$\epoch{0}$}
              +(225:0.7) node[op] (eA2) {$\epoch{A2}$}
              +(315:0.7) node[op, targetepoch] (eB3) {$\epoch{B3}$};

            \draw[->] (e0) edge (eA2) (eA2) (e0) edge (eB3);
            \draw[->, dashed, targetepoch] (eB3.180) -- (eA2.0) (eA2.0) -- (e0.270);
          \end{tikzpicture}
        }
      \end{figure}
    \end{column}
    \begin{column}{0.7 \textwidth}
      \begin{itemize}
        \item TODO: Illustrer choix de l'époque cible, cas 1 et cas 2
      \end{itemize}
    \end{column}
  \end{columns}
\end{frame}

% \begin{frame}{Downsides}
%   \begin{block}{Need to store former state until no more concurrent operations}
%     \begin{itemize}
%       \item Can garbage collect it once the \emph{rename} operation is causally stable \footcite{10.1007/978-3-662-43352-2_11}
%       \item Can offload it to the disk meanwhile
%     \end{itemize}
%   \end{block}

%   \begin{block}{Need to propagate former state to other nodes}
%     \begin{itemize}
%       \item Can compress the operation to minimise bandwidth consumption
%       \item Can trigger \emph{rename} operations at a given number of blocks
%     \end{itemize}
%   \end{block}
% \end{frame}
