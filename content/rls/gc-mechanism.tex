\subsection{Mécanisme de GC des époques obsolètes}

\begin{frame}[standout]
    \alert{Mais ça sert à quoi de renommer ?}

    Puisqu'on doit conserver les $\trm{renIds}$ pour gérer les opérations concurrentes\dots
\end{frame}

\begin{frame}{Suppression des $\trm{renIds}$}
    \vspace{-1.5em}
    \begin{columns}
        \begin{column}{0.7\textwidth}
            \begin{figure}
                \resizebox{\columnwidth}{!}{
                    \begin{tikzpicture}
                    \path
                        node {\textbf{A}}
                        ++(0:0.5) node (a) {}
                        +(0:8) node (a-end) {}
                        +(0:1) node (a-initial) {}
                        +(0:2) node[point, label=above:{$\trm{ren}(\epoch{0}, A,2,\renids{A2})$}] (a-ren-A2) {}
                        +(0:6) node[point] (a-recv-ren-B3) {}
                        +(0:7) node (a-final) {};

                    \draw[dotted] (a) -- (a-initial) (a-final) -- (a-end);

                    \path
                        ++(270:2) node {\textbf{B}}
                        ++(0:0.5) node (b) {}
                        +(0:8) node (b-end) {}
                        +(0:1) node (b-initial) {}
                        +(0:2) node[point, label=below:{$\trm{ren}(\epoch{0},B,3,\renids{B3})$}] (b-ren-B3) {}
                        +(0:6) node (b-ren-b7) {}
                        +(0:7) node (b-final) {};

                    \draw[dotted] (b) -- (b-initial) (b-final) -- (b-end);

                    \draw[->, thick] (a-initial) -- (a-ren-A2) -- (a-recv-ren-B3) -- (a-final);
                    \draw[->, dashed, shorten >= 1] (b-ren-B3) -- (a-recv-ren-B3);
                    \draw<1>[->, thick] (b-initial) -- (b-ren-B3) -- (b-final);
                    \onslide<2-> {
                        \path
                        (b-ren-b7) node[point, label=below:{$\trm{ren}(\epoch{B3},B,7,\renids{B7})$}] {};
                        \draw[->, thick] (b-initial) -- (b-ren-B3) -- (b-ren-b7) -- (b-final);
                    }
                    \end{tikzpicture}
                }
            \end{figure}
        \end{column}
        \begin{column}{0.3\textwidth}
            \begin{figure}
                \resizebox{\columnwidth}{!}{
                  \begin{tikzpicture}[scale=0.8,every node/.style={scale=0.3}]
                    \path
                      node[op] (e0) {$\epoch{0}$}
                      ++(225:0.7) node[op] (eA2) {$\epoch{A2}$};
                    \path
                      ++(315:0.7) node[op] (eB3) {$\epoch{B3}$}
                      ++(270:0.7) node (eB7) {};

                    \draw[->]
                        (e0) edge (eA2)
                        (e0) edge (eB3)
                    ;

                    \only<1>{
                        \path (eB3) node[label={[xshift=1em]right:{A,B}}] {};
                    }
                    \onslide<2->{
                      \path
                        (eB7) node[op] (eB7-node) {$\epoch{B7}$}
                        (eB7) node[label={[xshift=1em]right:{B}}] {}
                        (eB3) node[label={[xshift=1em]right:{A}}] {};

                      \draw[->] (eB3) edge (eB7-node);
                    }
                    \onslide<4->{
                        \draw[]
                            ([xshift=-0em,yshift=-0.3em]eA2.south west) -- ([xshift=0em, yshift=0.3em]eA2.north east)
                            ([xshift=-0em,yshift=-0.3em]e0.south west) -- ([xshift=0em, yshift=0.3em]e0.north east);
                    }
                  \end{tikzpicture}
                }
              \end{figure}
        \end{column}
    \end{columns}
    \begin{itemize}
        \item Besoin de garder $\trm{renIds}$ pour transformer opérations
        \item<3-> Si plus d'opérations nécessitant transformations vers époque donnée\dots
        \item<4-> \dots alors époque et $\trm{renIds}$ correspondant obsolètes
    \end{itemize}
    \vspace{-0.5em}
    \onslide<5>{
        \begin{block}{Besoins}
            \begin{itemize}
                \item \alert{Détecter stabilité causale}\footcite{10.1007/978-3-662-43352-2_11} des opérations \ren
                \item \alert{Connaître noeuds} appartenant au groupe
            \end{itemize}
        \end{block}
    }
\end{frame}
