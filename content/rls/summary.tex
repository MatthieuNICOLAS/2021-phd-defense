% \metroset{subsectionpage=progressbar}
\subsection{Synthèse}
% \metroset{subsectionpage=none}

\begin{frame}{Synthèse}
  \metroset{block=transparent}
  \begin{block}{Adaptation du mécanisme de renommage pour LogootSplit}
    \pause
    \begin{itemize}
      \item Opération \ren permettant de minimiser le surcoût de l'état
      \item Mécanisme de détection des opérations concurrentes
      \item Algorithme pour intégrer l'effet d'une opération \ren dans une opération \ins ou \rmv concurrente
    \end{itemize}
  \end{block}
  \pause
  \begin{block}{Conception d'un mécanisme de résolution de conflits pour opérations \ren concurrentes}
    \pause
    \begin{itemize}
      \item Mécanisme pour désigner une époque comme l'époque cible, sans coordination
      \item Algorithme pour annuler l'effet d'une opération \ren
    \end{itemize}
  \end{block}
  \pause
  \begin{block}{Conception d'un mécanisme de suppression des époques obsolètes}
  \end{block}
\end{frame}

% \begin{frame}{Conclusion}
%   \metroset{block=transparent}
%   \begin{block}{Contribution}
%     \begin{itemize}
%       \item Définition et validation mécanisme de renommage pour CRDTs pour Séquence à identifiants densément ordonnés, compatible avec systèmes pair-à-pair
%     \end{itemize}
%   \end{block}
%   \pause
%   \begin{block}{Limites}
%     \begin{itemize}
%       \item Surcoût fonction du nombre d'opérations \ren concurrentes
%       \item Stabilité causale \footcite{baquero2017pure} requise pour supprimer les métadonnées
%     \end{itemize}
%   \end{block}
%   \pause
%   \begin{block}{Perspectives}
%     \begin{itemize}
%       \item Stratégies de génération des opérations \ren
%       \item Relations \emph{priority} $\lepoch$ réduisant calculs à l'échelle du système
%       \item Preuve de correction de RLS
%     \end{itemize}
%   \end{block}
% \end{frame}
