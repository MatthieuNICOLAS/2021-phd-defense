\begin{frame}{Applications collaboratives}
    TODO: Voir comment représenter une appli collaboratice à ce stade
    \begin{itemize}
        \item Un \alert{système collaboratif} est un système supportant ses utilisateur-rices dans leurs processus de collaboration pour la réalisation de tâches.
    \end{itemize}
\end{frame}

\begin{frame}{Démocratisation des applications collaboratives}
    TODO: Voir comment illustrer ce point.
    Screenshots et nombre d'utilisateurs en-dessous ?
\end{frame}

\begin{frame}{Avantages d'une architecture basée sur le cloud...}
    TODO: Représenter collaboration via appli basée sur le cloud
    \begin{itemize}
        \item \alert{Disponibilité} : Répond aux utilisateur-rices
        \item \alert{Tolérance aux pannes} : Fonctionne malgré pannes
        \item \alert{Capacité de passage à l'échelle} : Supporte activité massive
    \end{itemize}
\end{frame}

\begin{frame}{... et ses limites}
    TODO: Illustrer chacune des propriétés
    \begin{columns}
        \begin{column}{0.5\textwidth}
            \begin{itemize}
                \item Confidentialité
                \item Souveraineté
            \end{itemize}
        \end{column}
        \begin{column}{0.5\textwidth}
            \begin{itemize}
                \item Pérennité
                \item Résistance à la censure
            \end{itemize}
        \end{column}
    \end{columns}
\end{frame}

\begin{frame}[standout]
    Pouvons-nous concevoir des applications collaboratives satisfaisant l'ensemble de ces propriétés ?
\end{frame}

\begin{frame}{Applications collaboratives pair-à-pair \footcite{localfirstsoftware2019}}
    TODO: Illustrer une appli P2P
    \begin{block}{Problématiques}
        En l'absence d'autorités centrales, comment
        \begin{itemize}
            \item résoudre les conflits de modifications ?
            \item authentifier les utilisateur-rices ?
            \item sécuriser les communications ?
        \end{itemize}
    \end{block}
\end{frame}


\begin{frame}{MUTE \singlefootnote{Disponible à : \url{https://mutehost.loria.fr}}}
    \begin{figure}
        \resizebox{0.8 \textwidth}{!}{
            \includegraphics{img/screenshot-mute-editor.png}
        }
    \end{figure}
    \begin{itemize}
        \item Éditeur de texte collaboratif P2P temps réel chiffré de bout en bout
        \item Permet à l'équipe d'étudier et contribuer sur les problématiques des applications \ac{LFS}
    \end{itemize}
\end{frame}

\begin{frame}{Mes contributions}
    TODO: À retravailler pour faire apparaître les problématiques ?
    TODO: Ajouter comparaison des modèles de sync et approches pour CRDTs pour le type Séquence ?
    \begin{itemize}
        \item Conception d'un nouveau \ac{CRDT} pour le type Séquence
        \item Implémentation et intégration de \acp{CRDT} dans MUTE
    \end{itemize}
\end{frame}
