\section{Problématiques des applications collaboratives}

\begin{frame}{MUTE \singlefootnote{Disponible à : \url{https://mutehost.loria.fr}}}
    \begin{figure}
        \resizebox{0.9 \textwidth}{!}{
            \includegraphics{img/screenshot-mute-editor.png}
        }
    \end{figure}
    \vspace{-0.5cm}
    \begin{itemize}
        \item Application pair-à-pair
        \item Permet à groupes de rédiger collaborativement documents texte
        \item Garantit confidentialité \& souveraineté de ses données
    \end{itemize}
\end{frame}

\begin{frame}{Applications collaboratives pair-à-pair}
    TODO: Illustrer une appli P2P
    \begin{block}{Problématiques}
        En l'absence d'autorités centrales, comment
        \begin{itemize}
            \item authentifier les utilisateur-rices ?
            \item vérifier leurs droits d'accès ?
            \item \alert{résoudre les conflits de modifications ?}
        \end{itemize}
    \end{block}
\end{frame}

\begin{frame}{Évaluation de MUTE}
    \metroset{block=transparent}
    \begin{block}{Taille du texte comparée à taille de la séquence répliquée}
        \begin{columns}
            \begin{column}{0.6\textwidth}
                \begin{figure}
                    \resizebox{\columnwidth}{!}{
                        \includegraphics{img/ls-vs-content-snapshot-sizes-7k5.pdf}
                    }
                \end{figure}
            \end{column}
            \begin{column}{0.4\textwidth}
                \begin{itemize}
                    \item 1\% contenu\dots
                    \item \dots 99\% métadonnées
                \end{itemize}
            \end{column}
        \end{columns}
    \end{block}
\end{frame}

\begin{frame}[standout]
    Comment peut-on réduire le surcoût mémoire des mécanismes de résolution de conflits dans les applications pair-à-pair ?
\end{frame}
