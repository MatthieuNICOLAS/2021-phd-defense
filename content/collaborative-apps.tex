\section{Problématiques des applications collaboratives}

\begin{frame}{MUTE \singlefootnote{Disponible à : \url{https://mutehost.loria.fr}}, un exemple de Local-First Software (LFS)\footcite{localfirstsoftware2019}}
    \vspace{-0.5cm}
    \begin{figure}
        \resizebox{\textwidth}{!}{
            \includegraphics{img/screenshot-mute-editor.png}
        }
    \end{figure}
    \vspace{-0.5cm}
    \begin{itemize}
        \item Application pair-à-pair
        \item Permet de rédiger collaborativement des documents texte
        \item Garantit la confidentialité \& souveraineté des données
    \end{itemize}
\end{frame}

\begin{frame}[fragile]{Réplication dans applications collaboratives pair-à-pair}
    \begin{figure}
        \resizebox{0.7 \textwidth}{!}{
            \begin{tikzpicture}
                \newcommand{\doc}{
                    \tikz{
                        \fill[scale=.15,fill=white,draw=gray,thick,solid] (0,0) -- (7,0) -- (7,8) -- (5,10) -- (0,10) -- cycle;
                    }
                }
                \newcommand{\updsquare}{
                    \tikz{
                        \fill[\colorblockone, scale=.12] (0,0) rectangle (3,3);
                    }
                }
                \newcommand{\updcircle}{
                    \tikz{
                        \fill[\colorblocktwo, scale=.07] (3,3) circle (3);
                    }
                }
                \newcommand{\updtriangle}{
                    \tikz{
                        \fill[\colorblockfive, scale=.07] (0,0) -- (6,0) -- (3,6) -- cycle;
                    }
                }
                \path
                    node[label=90:{A}] (a) {
                        \includegraphics[scale=0.4, page=5, trim=0cm 24cm 32cm 0cm, clip]{img/mute-figures.pdf}
                    }
                    +(-70:3) node[label=-90:{B}] (b) {
                        \includegraphics[scale=0.4, page=5, trim=0cm 24cm 32cm 0cm, clip]{img/mute-figures.pdf}
                    }
                    +(200:4) node[label=-90:{C}] (c) {
                        \includegraphics[scale=0.4, page=5, trim=0cm 24cm 32cm 0cm, clip]{img/mute-figures.pdf}
                    };

                \path
                    (a) node[label={[xshift=2em]0:{\doc}}] {}
                    (b) node[label={[xshift=2em]0:{\doc}}] {}
                    (c) node[label={[xshift=-2em]180:{\doc}}] {};

                \draw[dotted] (a) -- (b);

                \onslide<2->{
                    \path
                        (a) node[label={[xshift=3.8em]90:{\updsquare}}] {}
                        (b) node[label={[xshift=3.7em]0:{\updtriangle}}] {}
                        (c) node[label={[xshift=-4.8em]-90:{\updcircle}}] {};
                }

                \only<3>{
                    \path
                        (a) -- node[midway]{\includegraphics[scale=0.4]{img/sync.pdf}} (b);
                }

                \onslide<4->{
                    \path
                        (a) node[label={[xshift=3.7em]0:{\updtriangle}}] {}
                        (b) node[label={[xshift=3.8em]90:{\updsquare}}] {};
                }

                \onslide<5->{
                    \draw[dotted] (a) -- (c);
                    \draw[dotted] (b) -- (c);
                }

                \only<5> {
                    \path
                        (a) -- node[midway]{\includegraphics[scale=0.4]{img/sync.pdf}} (c)
                        (b) -- node[midway]{\includegraphics[scale=0.4]{img/sync.pdf}} (c);
                }

                \onslide<6->{
                    \path
                        (a) node[label={[xshift=3.8em]-90:{\updcircle}}] {}
                        (b) node[label={[xshift=3.8em]-90:{\updcircle}}] {}
                        (c) node[label={[xshift=-4.8em]90:{\updsquare}}] {}
                        (c) node[label={[xshift=-4.7em]0:{\updtriangle}}] {};
                }
            \end{tikzpicture}
        }
    \end{figure}
    \vspace{-1em}
    \begin{itemize}
        \item<7-> Doit garantir \alert{convergence à terme}\footcite{10.1145/224057.224070}\dots
        \item<7-> \dots malgré ordres différents d'intégration des modifications
    \end{itemize}
    \onslide<8>{
        \vspace{-1em}
        \begin{center}
            \alert{Nécessite des mécanismes de résolution de conflits}
        \end{center}
    }
\end{frame}

\begin{frame}{Évaluation de MUTE}
    \metroset{block=transparent}
    \begin{block}{Taille du texte comparée à taille de la séquence répliquée}
        \begin{columns}
            \begin{column}{0.6\textwidth}
                \begin{figure}
                    \resizebox{\columnwidth}{!}{
                        \includegraphics{img/ls-vs-content-snapshot-sizes-7k5.pdf}
                    }
                \end{figure}
            \end{column}
            \begin{column}{0.4\textwidth}
                \begin{block}{Constat}
                    \begin{itemize}
                        \item 1\% contenu\dots
                        \item \dots 99\% métadonnées
                    \end{itemize}
                \end{block}
                \pause
                \begin{center}
                    \alert{Et ça augmente !}
                \end{center}
                \pause
                \begin{block}{Impact}
                    \begin{itemize}
                        \item Surcoût \alert{mémoire}\dots
                        \item \dots mais aussi surcoût en \alert{calculs} et en \alert{bande-passante}
                    \end{itemize}
                \end{block}
            \end{column}
        \end{columns}
    \end{block}
\end{frame}

\begin{frame}[standout]
    Comment peut-on réduire le surcoût des mécanismes de résolution de conflits dans les applications pair-à-pair ?
\end{frame}
