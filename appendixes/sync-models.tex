\section{Modèles de synchronisation}

\begin{frame}{Doit-on encore concevoir CRDTs synchronisés par états ou opérations ?}
    \begin{table}[!ht]
      \resizebox{\columnwidth}{!}{
        \begin{tabular}{lccc}
          \toprule
                                                    & Sync. par états & Sync. par opérations    & Sync. par diff. d'états \\
          \midrule
          Forme un sup-demi-treillis                  & \checkmark  & \checkmark  & \checkmark  \\
          Intègre modifications par fusion d'états  & \checkmark  & \ballotx    & \checkmark  \\
          Intègre modifications par élts irréductibles & \ballotx    & \checkmark  & \checkmark  \\
          Résiste nativ. aux défaillances réseau           & \checkmark  & \ballotx    & \checkmark  \\
          Adapté pour systèmes temps réel           & \ballotx    & \checkmark  & \checkmark  \\
          Offre nativ. modèle de cohérence causale             & \checkmark  & \ballotx    & \ballotx  \\
          \bottomrule
        \end{tabular}
      }
    \end{table}
    \begin{itemize}
      \item Synchronisation par différences offre meilleur des mondes\dots
      \item \dots y a-t-il encore un intérêt aux autres modèles, \eg pour composition ou sécurité ?
    \end{itemize}
  \end{frame}
